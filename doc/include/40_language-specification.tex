\part{Language Specification}
\chapter{Design Principles}
<<<<<<< HEAD
The type system used for \ooplss is similar to that of \emph{TooL}
\cite{gawecki_tool:_1995}. Anyway, many important aspects differ from
\emph{Tool}. The used principles are these:

\paragraph{Purely object-oriented}
\ooplss is in contrast to its target platform Java a purely
object-oriented language. Every program entity is represented by an
object that can receive messages from other objects. There exists no
primitive data types which are not objects. Computing is performed only
by message passing between objects.

\paragraph{Class Based}
Creating new objects are an essential part in object-oriented
programming. In \ooplss classes are

\paragraph{Static Type System}
To reduce the set of possible runtime exceptions \ooplss is statically
and strongly typed. Every message sent to an object is legal is possible
and can not create exceptions on runtime. Nonetheless runtime exceptions
can occur since the type system does not prove that every statement is
possible, e.g., sending a division by zero message to a number object,
an appropriate exception will be thrown.

\paragraph{Safe Type System}

\paragraph{Nominal Type System}
There are two options how to determine the relation of subtyping
respectively matching between two types. On one hand is structural
analysis of the given types in the current context and on the other
is the analysis of the type hierarchy and restricting subtyping and
subclassing. Each concept has its assets and drawbacks but both can
be provided simultaneously \cite{malayeri_integrating_2008}. However,
\emph{TooL} and \emph{PolyTOIL} use structural subtyping compared to
Java which uses nominal subtyping. \ooplss has a nominal type system
for a modular type checking. Once the inheritance relations are checked,
it is easy to determine whether two types are in relation or not. Since
\ooplss is translated to Java a similar system is used, i.e., a nominal
type system is implemented.

\paragraph{No Information Hiding}
Since nominal subtyping is used, information hiding has not that impact
compared to structural subtyping since all fields are automatically
available. This helps to keep the language as small.
% estimated pages: 2
=======
\cite{gawecki_tool:_1995}
\paragraph{Purely object-oriented}
\paragraph{Class based}
\paragraph{Static type system}
\paragraph{Nominal subtyping}
\paragraph{Structural matching}

% paragraph paragraph name (end)
% estimated pages: 2-3
>>>>>>> Doc: Binary Methods
% summary of language design

\chapter{Abstract Syntax}
The language foundation of \ooplss is kept simple and compact. 

\section{Syntax}
\begin{listing}
	\begin{tabular}[H]{llrll}
		\emph{Class}					& C	& $\longrightarrow$ 		& A[B] Subclass Of D[T], E[U] Subtype Of F[V] \{T a; T M;\} \\
		\emph{Types}					& T & $\longrightarrow$ 		& A \\
													&   & 									 | & A \match B[A] \\
		\emph{Method}					& M	& $\longrightarrow$ 		& T m[S](U a, V b) \{e;\} \\
		\emph{Constructor}		& K	& $\longrightarrow$ 		& c(U a, V b) \{A(a); e;\} \\
		\emph{Expression}			& e & $\longrightarrow$ 		& x; \\
													&   & 									 | & x = y; \\
													&   & 									 | & e.f; \\
													&   & 									 | & e.m(p); \\
													&   & 									 | & new C(p); \\
	\end{tabular}
\caption{Abstract syntax of \ooplss}
\label{lst:abstractSyntax}
\end{listing}

% estimated pages: 10

\chapter{Names and Scopes}
% estimated pages: 2

\chapter{Types}
% estimated pages: 14
\section{Basic Types}
\section{Subtypes}
\section{Subclasses}

\chapter{Classes}
% estimated pages: 6

\chapter{Syntax and Type Inference}
% estimated pages:	5
