\part{Project Management}
\chapter{Methodologies}
Many different software development processes have risen since the
breakthrough of software developing. Two major trends can be recognised;
Sequential and iterative design processes. Although iterative processes
have many advantages compared to the others, it is difficult to follow
strictly this design method. On one hand the development time is very
short for regular iterations on the other hand the project includes
many non standard design and development parts where such a process is
difficult to perform. E.g. extending subsequently the grammar may end
in major changes in a language. However, a not very strict iterative
model is chosen.

\section{Iterative and Agile Development}
This project was build iteratively in the sense of continuous
refinement in the definition of the language. The language definition and
implementation goes hand in hand and gives direct feedback. This makes it
possible to test the definition on a concrete implementation. Further
parts from extreme programming are used. Manly the concept of pair
programming for critical parts which provide interfaces between the
compiler stages.

\section{Test-Driven Development}
A major goal is to provide a good and robust code basis. To ensure that
this claim holds, test-driven development is performed. With the use of
unit tests code breaking changes can be easily detected.

\section{Regular Meetings}
The Bern University of Applied Science does not provide a major in
topics used in this thesis. To ensure success regular meetings with the
supervisors are hold. With this early divergence and loss of target should
be prevented. A meeting between the students team and the supervisor
should be at least every two weeks.

\chapter{Project Schedule}
\section{Tasks}

\section{Diagram}
\begin{gantt}[xunitlength=0.7cm]{12}{16}
	\begin{ganttitle}
		\numtitle{1}{1}{16}{1}
	\end{ganttitle}
	\ganttbar{Kick-Off}{0}{2}
	\ganttbarcon{Documentation Part I / II}{2}{3}
	\ganttbar{Grammar / Lexer}{3}{2}
	\ganttbarcon{Types}{5}{2}
	\ganttbar{Parser}{5}{3} % Con
	\ganttbar{Tree Walking}{8}{1}
	\ganttbarcon{Symbol Table Construction}{9}{1}
	\ganttbar{Type Checker}{9}{6} % Con
	\ganttbar{Code Translation Definition}{6}{2}
	\ganttbarcon{Code Translation}{11}{5}
	\ganttbar{Documentation}{0}{16}

	\ganttcon{5}{3}{5}{5}
	\ganttcon{8}{5}{8}{6}
	\ganttcon{8}{5}{8}{6}
	\ganttcon{9}{6}{9}{8}
\end{gantt}

\chapter{Tools and Software}
\begin{description}
	\item[pdf\LaTeX] 
	\item[Subversion]
	\item[Git]
	\item[ANTLR3]
	\item[Java$^{\textrm{TM}}$]
	\item[JUnit]
	\item[Eclipse]
	\item[ant]
\end{description}

\chapter{Standards and Guidelines}
Coding guidelines

\chapter{Journal}
\begin{center}
\begin{longtable}{|c|p{12cm}|}

\hline \multicolumn{1}{|c|}{\textbf{Week}} & \multicolumn{1}{c|}{\textbf{Journal Entry}} \\ \hline 
\endfirsthead

\hline \multicolumn{1}{|c|}{\textbf{Week}} &
\multicolumn{1}{c|}{\textbf{Journal Entry}} \\
\endhead

\hline \multicolumn{2}{|r|}{{Continued on next page}} \\ \hline
\caption[Journal]{Journal} \label{table:journal} \\
\endfoot

\multicolumn{2}{c}%
{{\bfseries \tablename\ \thetable{} -- continued from previous page}} \\
\endlastfoot

1 & 
\begin{journal}
	\item Project structure defined
	\item Repository installed
	\item Major task identified
\end{journal}
\\ \hline
2 & 
\begin{journal}
	\item Documented the theoretical background
\end{journal}
\\ \hline
3 & 
\begin{journal}
	\item Documented the theoretical background
\end{journal}
\\ \hline
4 & 
\begin{journal}
	\item Documented the theoretical background
	\item Defined a basic draft of the language's syntax
	\item Decision between JavaCC and ANTLR -> ANTLR chosen
\end{journal}
\\ \hline
5 & 
\begin{journal}
	\item Started the ANTLR grammar
	\item Writing gUnit tests for the grammar
\end{journal}
\\ \hline
6 & 
\begin{journal}
	\item Extended the ANTLR grammar
	\item Started to write AST rewrite rules
\end{journal}
\\ \hline
7 & 
\begin{journal}
	\item 
\end{journal}
\\ \hline
8 & 
\begin{journal}
	\item 
\end{journal}
\\ \hline
9 & 
\begin{journal}
	\item 
\end{journal}
\\ \hline
10 & 
\begin{journal}
	\item 
\end{journal}
\\ \hline
11 & 
\begin{journal}
	\item 
\end{journal}
\\ \hline
12 & 
\begin{journal}
	\item 
\end{journal}
\\ \hline
13 & 
\begin{journal}
	\item 
\end{journal}
\\ \hline
14 & 
\begin{journal}
	\item 
\end{journal}
\\ \hline
15 & 
\begin{journal}
	\item 
\end{journal}
\\ \hline
16 & 
\begin{journal}
	\item 
\end{journal}
\\ \hline
\end{longtable}
\end{center}
