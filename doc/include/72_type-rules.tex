\chapter{Type Rules}
\label{ctr:typeRules}
\section{Object Generation}
\section{Higher-Order Subtyping}
The following definitions for \emph{Oper} and \emph{Type} relation are
taken form \cite{steffen_higher-order_1994}.

\begin{tabular}{lllr}
	\oper{X} & $\approx$ & X & \emph{(assuming that X is match-bound)} \\
	\oper{$\mu X.\{\overline{l:T}, \overline{m:U\{X\}}\}$}
		& $\approx$
		& \multicolumn{2}{l}{$\lambda X.\{\overline{l:\text{\type{T}}}, \overline{m:\text{\type{U\{X\}}}}\}$} \\
	\type{X} & $\approx$ & X$^*$ & \emph{(when X is match-bound)} \\
	\type{$\mu X.\{\overline{l:T}, \overline{m:U\{X\}}\}$}
		& $\approx$
		& \multicolumn{2}{l}{$\mu X.\{\overline{l:\text{\type{T}}}, \overline{m:\text{\type{U\{X\}}}}\}$} \\
	\type{X} & $\approx$ & X & \emph{(when X is not match-bound)} \\
	\type{A $\rightarrow$ B} & $\approx$ & \multicolumn{2}{l}{\type{A} $\rightarrow$ \type{B}} \\
	\type{$\forall(X \mmatch A)B$} & $\approx$ & \multicolumn{2}{l}{$\forall(X \prec: \text{\oper{A}})$ \type{B}} \\
	A \match B & $\approx$ & \oper{A} $ \prec:$ \oper{B} & \emph{(for A, B object types)} \\
\end{tabular}

Using these definitions implies reflexivity and transitivity for matching since this is true for subtyping.

\section{Type Generation}

\section{Type Rules}
For the sake of comprehensibility the following convention is used within the type rules described in this section.

\begin{description}
	\item[Type Variables] $X, Y, Z$
	\item[Type Arguments] $T, U, V, W$
	\item[Constructor Type] $K$
	\item[Object Types] $A - J \ F$
	\item[Method Types] $M - P$
	\item[Field Type] $F$
	\item[Type Boundary] $N$
\end{description}

For more compactness the following abbreviations and symbols are used; the keywords \emph{subtypeOf} and \emph{subclassOf} are expressed with $\subt$ and $\subc$ $\subcClosure$ $\subtClosure$

\begin{figure}[H]
 \fbox{
    \begin{minipage}{\linewidth}
			\paragraph{Subtyping:}
				\begin{mathpar}
					\inferrule*[right=SubT-Ref,rightskip=-1.0cm]{\\}{A <: A}
					\inferrule*[right=SubT-Trans]{B <: A \\ C <: B}{C <: A}
				\end{mathpar}

			\paragraph{Subclassing:}

			\paragraph{Matching:}
				\begin{mathpar}
					\inferrule*[right=Match-Ref,rightskip=-1.0cm]{\\}{A \mmatch A}
					\inferrule*[right=Match-Trans]{B \mmatch A \\ C \mmatch B}{C \mmatch A} \\
					\inferrule*[right=Match-Adopt,rightskip=-1.0cm]{B <: A \\ C \mmatch B}{C \mmatch A}
					\inferrule*[right=Match-Adopt]{B <: A \\ C \mmatch B}{B \mmatch A} \\
					\inferrule*[right=Match-Embed-SubT]{B <: A}{B \mmatch A}
				\end{mathpar}
    \end{minipage}
  }
	\caption{foo}
	\label{fig:rule}
\end{figure}

\begin{itemize}
\item Product Introduction
\item Product Elimination
\item Function Introduction
\item Function Elimination
\item Record Introduction
\item Record Elimination
\item Record Subtyping
\item Function Subtyping
\item Product Subtyping
\item Record Extension
\item Record Overriding
\item Recursive Subtype
\item Subclass
\item Classify
\end{itemize}
