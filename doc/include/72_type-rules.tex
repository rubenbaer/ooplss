\chapter{Type Inference Rules}
\label{ctr:typeRules}
\section{Type Rules}
For the sake of comprehensibility the following convention is used within
the type rules described in this section:

\begin{table}[H]
	\begin{tabular}{ll}
		\textit{Type Variables} & $X, Y, Z$ \\
		\textit{Type Arguments} & $T, U, V, W$ \\
		\textit{Constructor Type} & $K$ \\
		\textit{Object Types} & $A - E, G - J$ \\
		\textit{Field Type} & $F$ \\
		\textit{Method Types} & $M - P$ \\
		\textit{Type Boundary} & $N$
	\end{tabular}
\end{table}

For more compactness the following abbreviations and symbols are used;
the keywords \emph{subtypeOf} and \emph{subclassOf} are expressed
with $\subt$ and $\subc$, respectively. The reflexive and transitive
closure for subtyping and subclassing is denoted by $\subtc$ and
$\underline{\blacktriangleleft}$, respectively.  To specify a list of
length $0\ldots n$ an over line is used, e.g., $\overline{X \mmatch N}$
is a list of match bounds.

Since type parameters are used, two environments are introduced. The
notion used within the type inference rules are similar to them from
\emph{Featherweight Java}\nomenclature{Featherweight Java}{A minimal
core calculus for Java and GJ}. `An environment $\Gamma$ is a finite
mapping from variables to types, written $\overline{x}:\overline{T}$;
a type environment $\Delta$ is a finite mapping from type variables to
nonvariable types, wirtten $\overline{X}<:\overline{N}$, which takes
each type variable to its bound' \cite{igarashi_featherweight_1999}. In
contrast to Featherweight Java, the bound is specified by \match and
using a type operator.

It is assumed that the implicit binding of \mytype is done as described
in \Cref{sec:implicitlyMyType} but made explicit, i.e., all necessary
parameters are introduced by the type combination to bind \mytype
correctly.

The typing rules for special constructs like \emph{if} or \emph{while}
statements are not stated. They can straightforwardly be implemented
using the expression rules.

\begin{figure}[H]
 \fbox{
    \begin{minipage}{\linewidth-0.5cm}
			\paragraph{Subclassing:}
				\begin{mathpar}
					\inferrule*[right=SubC-Ref,rightskip=-1.0cm]
						{\\}
						{A \subcc A}
					\inferrule*[right=SubC-Trans]
						{B \subcc A\\C \subcc B}
						{C \subcc A} \\
					\inferrule*[right=SubC-Intro]
						{\text{class }A[\overline{X \mmatch N}] \subc B[\overline{T}] \{\ldots\}}
						{A \subcc B}
				\end{mathpar}

			\paragraph{Subtyping:}
				\begin{mathpar}
					\inferrule*[right=SubT-Ref,rightskip=-1.0cm]
						{\\}
						{A \subtc A}
					\inferrule*[right=SubT-Trans]
						{B \subtc A \\ C \subtc B}
						{C \subtc A} \\
					\inferrule*[right=SubT-Intro]
						{\text{class }A[\overline{X \mmatch N}] \subt B[\overline{T}] \{\ldots\}}
						{A \subtc B}
				\end{mathpar}
    \end{minipage}
  }
	\caption{Type closures.}
	\label{fig:basicRules}
\end{figure}

\begin{figure}[H]
 \fbox{
    \begin{minipage}{\linewidth-0.5cm}
			\paragraph{Field lookup:}
				\begin{mathpar}
					\inferrule*[right=F-Empty]
						{\text{class }A[X \mmatch N] \{ \}}
						{fields(A[T]) = \emptyset} \\
					\inferrule*[right=F-Class]
						{\text{class } A[\overline{X \mmatch N}]
								\subt B[\overline{T}]
								\subc C[\overline{U}] \{\overline{S~f}; K~ \overline{M}\} \\\\
							fields([\overline{T / X}]B)=\overline{U~g} \\\\
							fields([\overline{T / X}]C)=\overline{V~h}}
						{fields(A[\overline{T}]) = \overline{U~g}, \overline{V~h}, [\overline{T / X}]\overline{S~f} }
				\end{mathpar}

			\paragraph{Method type lookup:}
				\begin{mathpar}
					\inferrule*[right=MT-Class]
						{\text{class } A[\overline{X \mmatch N}]
								\subt B[\overline{T}]
								\subc C[\overline{U}] \{\overline{S~f}; K~ \overline{M}\} \\\\
							\text{def } m(\overline{U~x}): R \{ \ldots \} \in \overline{M}}
						{mtype(m, A[\overline{T}]) = [\overline{T/X}](\overline{U} \rightarrow R)} \\

					\inferrule*[right=MT-Supertype]
						{\text{class } A[\overline{X \mmatch N}]
								\subt B[\overline{T}]
								\subc C[\overline{U}] \{\overline{S~f}; K~ \overline{M}\} \\\\
							m \in B[\overline{T}] \\ m \notin \overline{M}}
						{mtype(m, A[\overline{T}]) = mtype(m, [\overline{T/X}]B)}\\

					\inferrule*[right=MT-Superclass]
						{\text{class } A[\overline{X \mmatch N}]
								\subt B[\overline{T}]
								\subc C[\overline{U}] \{\overline{S~f}; K~ \overline{M}\} \\\\
							m \in C[\overline{T}] \\ m \notin \overline{M}}
						{mtype(m, A[\overline{T}]) = mtype(m, [\overline{T/X}]C)}\\
				\end{mathpar}

			\paragraph{Method body lookup:}
				\begin{mathpar}
					\inferrule*[right=MT-Class]
						{\text{class } A[\overline{X \mmatch N}]
								\subt B[\overline{T}]
								\subc C[\overline{U}] \{\overline{S~f}; K~ \overline{M}\} \\\\
							\text{def } m(\overline{U x}): R \{ e_i^n (n \in \mathbb{N}) \} \in \overline{M}}
						{mbody(m, A[\overline{T}]) = \overline{X}.[\overline{T/X}].e_i^n (n \in \mathbb{N})} \\

					\inferrule*[right=MT-Supertype]
						{\text{class } A[\overline{X \mmatch N}]
								\subt B[\overline{T}]
								\subc C[\overline{U}] \{\overline{S~f}; K~ \overline{M}\} \\\\
							m \in B[\overline{T}] \\ m \notin \overline{M}}
						{mbody(m, A[\overline{T}]) = mbody(m, [\overline{T/X}]B)}\\

					\inferrule*[right=MT-Superclass]
						{\text{class } A[\overline{X \mmatch N}]
								\subt B[\overline{T}]
								\subc C[\overline{U}] \{\overline{S~f}; K~ \overline{M}\} \\\\
							m \in C[\overline{T}] \\ m \notin \overline{M}}
						{mbody(m, A[\overline{T}]) = mbody(m, [\overline{T/X}]C)}\\
				\end{mathpar}

			\paragraph{Bound of types:}
					\[bound_\Delta(X) = \Delta(X) \]
					\[bound_\Delta(A) = A \]
    \end{minipage}
  }
	\caption{Auxiliary functions.}
	\label{fig:auxFunctions}
\end{figure}

\begin{figure}[H]
 \fbox{
    \begin{minipage}{\linewidth-0.5cm}
			\paragraph{Matching:}
				\begin{mathpar}
					\inferrule*[right=Subclass-Ref,rightskip=-1.0cm]
						{\\}
						{\Delta \vdash A \mmatch A}
					\inferrule*[right=Match-Trans]
						{\Delta \vdash B \mmatch A \\ \Delta \vdash C \mmatch B}
						{\Delta \vdash C \mmatch A} \\
					\inferrule*[right=Match-Adopt,rightskip=-1.0cm]
						{\Delta \vdash B <: A \\ \Delta \vdash C \mmatch B}
						{\Delta \vdash C \mmatch A}
					\inferrule*[right=Match-Embed-SubT]
						{\Delta \vdash B <: A}
						{\Delta \vdash B \mmatch A} \\
					\inferrule*[right=Match-Var]
						{\\}
						{\Delta \vdash X \mmatch \Delta(X)} \\
					\inferrule*[right=Match-Class]
						{\text{class }X[\overline{X \mmatch N}] \subt B \subc C \{\ldots\}}
						{\Delta \vdash A[\overline{T}] \mmatch [\overline{T/X}]B\\
						\Delta \vdash A[\overline{T}] \mmatch [\overline{T/X}]C}
				\end{mathpar}

			\paragraph{Subtyping:}
				\begin{mathpar}
					\inferrule*[right=Subtype-Ref,rightskip=-1.0cm]
						{\\}
						{\Delta \vdash A <: A}
					\inferrule*[right=Subtype-Trans]
						{\Delta \vdash B <: A \\ \Delta \vdash C <: B}
						{\Delta \vdash C <: A} \\
					\inferrule*[right=Subtype-Class]
						{\text{class }X[\overline{X \mmatch N}] \subt B \subc C \{\ldots\}}
						{\Delta \vdash A[\overline{T}] <: [\overline{T/X}]B}
				\end{mathpar}
		\end{minipage}
	}
	\caption{Type relation rules.}
	\label{fig:typeRelationsRules}
\end{figure}

\begin{figure}[H]
 \fbox{
    \begin{minipage}{\linewidth-0.5cm}
			\paragraph{Well-formed types:}
				\begin{mathpar}
					\inferrule*[right=WF-Env,rightskip=-1.0cm]
						{\\}
						{\Delta \vdash ok}
					\inferrule*[right=WF-VarType]
						{X \in dom(\Delta)}
						{\Delta \vdash X~ok}\\
					\inferrule*[right=WF-Class]
						{\text{class }A[\overline{X \mmatch N}]
								\subt B
								\subc C \{\ldots\}\\\\
							\Delta \vdash \overline{T}~ok \\
							\Delta \vdash \overline{T \mmatch [\overline{T/X}]N}
						}
						{\Delta \vdash C[\overline{T}]~ok}
				\end{mathpar}
		\end{minipage}
	}
	\caption{Wellformed type rules.}
	\label{fig:wellformed}
\end{figure}

\begin{figure}[H]
 \fbox{
    \begin{minipage}{\linewidth-0.5cm}
			\paragraph{Expression typing:}
				\begin{mathpar}
					\inferrule*[right=Var]
						{\\}
						{\Delta; \Gamma \vdash x : \Gamma(x)} \\
					\inferrule*[right=Field]
						{\Delta; \Gamma \vdash e_i : T_i \\ fields(bound_\Delta(T_i)) = \overline{T~f}}
						{\Delta; \Gamma \vdash e_i.f_j : T_j} \\
					\inferrule*[right=Invk]
						{\Delta; \Gamma \vdash e_i : T_i \\\\
							mtype(m, bound_\Delta(T_i)) = \overline{U} \rightarrow U \\\\
							\Delta; \Gamma \vdash \overline{e : S} \\ \Gamma \vdash \overline{S <: U}}
						{\Delta; \Gamma \vdash e_i.m(\overline{e}) : U} \\
					\inferrule*[right=New]
						{\Delta \vdash N~ok \\
							fields(N) = \overline{T f} \\
							\Delta; \Gamma \vdash \overline{e : S} \\
							\Delta \vdash \overline{S <: T}}
						{\Delta; \Gamma \vdash \text{new }N(\overline{e}) : N}
				\end{mathpar}
		\end{minipage}
	}
	\caption{Expression type rules.}
	\label{fig:expressionRuleexpressionRules}
\end{figure}

\begin{figure}[H]
 \fbox{
    \begin{minipage}{\linewidth-0.5cm}
			\paragraph{Method typing:}
				\begin{mathpar}
					\inferrule*[right=Method]
						{\Delta; \overline{x:T}; \text{self}:C[\overline{T}] \vdash e_n : S \\
							\Delta \vdash S <: T \\
							\text{class } A[\overline{X \mmatch N}] \subt B \subc C \{\ldots\} \\
							(m : \overline{T} \rightarrow T) \in (B \cup C)}
						{\text{def }m(\overline{T x}):U \{\overline{e}; \text{return } e_n;\}~ok~in~A[\overline{X \mmatch N}]}
				\end{mathpar}

			\paragraph{Class typing:}
				\begin{mathpar}
					\inferrule*[right=Class]
						{\overline{X \mmatch N} \vdash \overline{B}, B, \overline{C}, C, \overline{T}~ok\\
							fields(B) = \overline{U~g}\\
							fields(C) = \overline{V~h}\\
							\overline{M}~ok~in~A[\overline{X \mmatch N}] \\
							K=A(\overline{U~g}, \overline{V~h}, \overline(T~f)): B(\overline(g), C(\overline{h}) \{ \ldots \}}
						{class A[\overline{X \mmatch N}] \subt B \subc C \{\overline{U f}; K~\overline{M}\}}
				\end{mathpar}
		\end{minipage}
	}
	\caption{Method and Class type rules.}
	\label{fig:methodClassRules}
\end{figure}
