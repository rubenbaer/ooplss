\chapter{Type Inference Rules}
\label{ctr:typeRules}
\section{Object Generation}
\section{Higher-Order Subtyping}
The following definitions for \emph{Oper} and \emph{Type} relation are
taken form \cite{steffen_higher-order_1994}.
\begin{table}[H]
	\begin{tabular}{lllr}
		\oper{X} & $\approx$ & X & \emph{(when X is match-bound)} \\
		\oper{$\mu X.\{\overline{l:T}, \overline{m:U\{X\}}\}$}
			& $\approx$
			& \multicolumn{2}{l}{$\lambda X.\{\overline{l:\text{\type{T}}}, \overline{m:\text{\type{U\{X\}}}}\}$} \\
		\type{X} & $\approx$ & X$^*$ & \emph{(when X is match-bound)} \\
		\type{$\mu X.\{\overline{l:T}, \overline{m:U\{X\}}\}$}
			& $\approx$
			& \multicolumn{2}{l}{$\mu X.\{\overline{l:\text{\type{T}}}, \overline{m:\text{\type{U\{X\}}}}\}$} \\
		\type{X} & $\approx$ & X & \emph{(when X is not match-bound)} \\
		\type{A $\rightarrow$ B} & $\approx$ & \multicolumn{2}{l}{\type{A} $\rightarrow$ \type{B}} \\
		\type{$\forall(X \mmatch A)B$} & $\approx$ & \multicolumn{2}{l}{$\forall(X \prec: \text{\oper{A}})$ \type{B}} \\
		A \match B & $\approx$ & \oper{A} $ \prec:$ \oper{B} & \emph{(for A, B object types)} \\
	\end{tabular}
\end{table}
Using these definitions implies reflexivity and transitivity for matching
since this is true for subtyping.
%
%\section{Type Generation}

\section{Type Rules}
For the sake of comprehensibility the following convention is used within
the type rules described in this section:

\begin{table}[H]
	\begin{tabular}{ll}
		\textit{Type Variables} & $X, Y, Z$ \\
		\textit{Type Arguments} & $T, U, V, W$ \\
		\textit{Constructor Type} & $K$ \\
		\textit{Object Types} & $A - E, G - J$ \\
		\textit{Field Type} & $F$ \\
		\textit{Method Types} & $M - P$ \\
		\textit{Type Boundary} & $N$
	\end{tabular}
\end{table}

For more compactness the following abbreviations and symbols are used; the
keywords \emph{subtypeOf} and \emph{subclassOf} are expressed with $\subt$
and $\subc$. The reflexive and transitive closure for subtyping and
subclassing is denoted by $\subtc$ and $\underline{\blacktriangleleft}$.
To specify a list of length $0\ldots n$ a over line is used, e.g.,
$\overline{X \mmatch N}$ is a list of match bounds.

TODO:\todo{TODO}
It is assumed that the implicit binding of \mytype is done as described in \Cref{sec:implicitlyMyType} but made explicit, i.e., all necessary parameters are introduced by the type combination to bind \mytype correctly.

%$\subcClosure$ $\subtClosure$
\begin{figure}[H]
 \fbox{
    \begin{minipage}{\linewidth}
			\paragraph{Subclassing:}
				\begin{mathpar}
					\inferrule*[right=SubC-Ref,rightskip=-1.0cm]{\\}{A \subcc A}
					\inferrule*[right=SubC-Trans]{B \subcc A\\C \subcc B}{C \subcc A} \\
					\inferrule*[right=SubC-Intro]{\text{class }A[\overline{X \mmatch N}] \subc B[\overline{T}] \{\ldots\}}{A \subcc B}
				\end{mathpar}

			\paragraph{Subtyping:}
				\begin{mathpar}
					\inferrule*[right=SubT-Ref,rightskip=-1.0cm]{\\}{A \subtc A}
					\inferrule*[right=SubT-Trans]{B \subtc A \\ C \subtc B}{C \subtc A} \\
					\inferrule*[right=SubT-Intro]{\text{class }A[\overline{X \mmatch N}] \subt B[\overline{T}] \{\ldots\}}{A \subtc B}
				\end{mathpar}
    \end{minipage}
  }
	\caption{Basic rules}
	\label{fig:basicRules}
\end{figure}

\begin{figure}[H]
 \fbox{
    \begin{minipage}{\linewidth}
			\paragraph{Field lookup:}
				\begin{mathpar}
					\inferrule*[right=F-Empty]{\\}{fields(\emptyset) = \emptyset} \\
					\inferrule*[right=F-Class]
						{\text{class } A[\overline{X \mmatch N}]
								\subt B[\overline{T}]
								\subc C[\overline{U}] \{\overline{S f}; K \overline{M}\} \\\\
							fields([\overline{T / X}]B)=\overline{U g} \\\\
							fields([\overline{T / X}]C)=\overline{V h}}
						{fields(A[\overline{T}]) = \overline{U g}, \overline{V h}, [\overline{T / X}]\overline{S f} }
				\end{mathpar}

			\paragraph{Method type lookup:}
				\begin{mathpar}
					\inferrule*[right=MT-Class]
						{\text{class } A[\overline{X \mmatch N}]
								\subt B[\overline{T}]
								\subc C[\overline{U}] \{\overline{S f}; K \overline{M}\} \\\\
							\text{def } m(\overline{U x}): R \{ \ldots \} \in \overline{M}}
						{mtype(m, A[\overline{T}]) = [\overline{T/X}](\overline{U} \rightarrow R)} \\

					\inferrule*[right=MT-Supertype]
						{\text{class } A[\overline{X \mmatch N}]
								\subt B[\overline{T}]
								\subc C[\overline{U}] \{\overline{S f}; K \overline{M}\} \\\\
							m \in B[\overline{T}] \\ m \notin \overline{M}}
						{mtype(m, A[\overline{T}]) = mtype(m, [\overline{T/X}]B)}\\

					\inferrule*[right=MT-Superclass]
						{\text{class } A[\overline{X \mmatch N}]
								\subt B[\overline{T}]
								\subc C[\overline{U}] \{\overline{S f}; K \overline{M}\} \\\\
							m \in C[\overline{T}] \\ m \notin \overline{M}}
						{mtype(m, A[\overline{T}]) = mtype(m, [\overline{T/X}]C)}\\
				\end{mathpar}

			\paragraph{Method body lookup:}
				\begin{mathpar}
					\inferrule*[right=MT-Class]
						{\text{class } A[\overline{X \mmatch N}]
								\subt B[\overline{T}]
								\subc C[\overline{U}] \{\overline{S f}; K \overline{M}\} \\\\
							\text{def } m(\overline{U x}): R \{ e_i^n (n \in \mathbb{N}) \} \in \overline{M}}
						{mbody(m, A[\overline{T}]) = \overline{X}.[\overline{T/X}].e_i^n (n \in \mathbb{N})} \\

					\inferrule*[right=MT-Supertype]
						{\text{class } A[\overline{X \mmatch N}]
								\subt B[\overline{T}]
								\subc C[\overline{U}] \{\overline{S f}; K \overline{M}\} \\\\
							m \in B[\overline{T}] \\ m \notin \overline{M}}
						{mbody(m, A[\overline{T}]) = mbody(m, [\overline{T/X}]B)}\\

					\inferrule*[right=MT-Superclass]
						{\text{class } A[\overline{X \mmatch N}]
								\subt B[\overline{T}]
								\subc C[\overline{U}] \{\overline{S f}; K \overline{M}\} \\\\
							m \in C[\overline{T}] \\ m \notin \overline{M}}
						{mbody(m, A[\overline{T}]) = mbody(m, [\overline{T/X}]C)}\\
				\end{mathpar}
    \end{minipage}
  }
	\caption{Auxiliary functions}
	\label{fig:auxFunctions}
\end{figure}

\begin{figure}[H]
 \fbox{
    \begin{minipage}{\linewidth}
			\paragraph{Matching:}
				\begin{mathpar}
					\inferrule*[right=Match-Ref,rightskip=-1.0cm]{\\}{\Delta \vdash A \mmatch A}
					\inferrule*[right=Match-Trans]{B \mmatch A \\ C \mmatch B}{C \mmatch A} \\
					\inferrule*[right=Match-Adopt,rightskip=-1.0cm]{B <: A \\ C \mmatch B}{C \mmatch A}
					\inferrule*[right=Match-Embed-SubT]{B <: A}{B \mmatch A}
				\end{mathpar}

			\paragraph{Subtyping:}
				\begin{mathpar}
					\inferrule*[right=Subtype-Ref,rightskip=-1.0cm]{\\}{A <: A}
					\inferrule*[right=Subtype-Trans]{B <: A \\ C <: B}{C <: A} \\
					\inferrule*[right=Subtype-Adopt]{B <: A \\ C <: B}{C <: A}
				\end{mathpar}
		\end{minipage}
	}
	\caption{Type relation rules}
	\label{fig:typeRelationsRules}
\end{figure}

\begin{figure}[H]
 \fbox{
    \begin{minipage}{\linewidth}
		\end{minipage}
	}
	\caption{Subclassing type rules}
	\label{fig:subclassingRules}
\end{figure}

\begin{figure}[H]
 \fbox{
    \begin{minipage}{\linewidth}
		\end{minipage}
	}
	\caption{Subtyping type rules}
	\label{fig:subtypingRules}
\end{figure}

%\begin{itemize}
%\item Product Introduction
%\item Product Elimination
%\item Function Introduction
%\item Function Elimination
%\item Record Introduction
%\item Record Elimination
%\item Record Subtyping
%\item Function Subtyping
%\item Product Subtyping
%\item Record Extension
%\item Record Overriding
%\item Recursive Subtype
%\item Subclass
%\item Classify
%\end{itemize}
