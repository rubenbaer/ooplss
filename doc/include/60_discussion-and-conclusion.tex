
\chapter{Discussion and Conclusion}
\label{ctr:discussionConclusion}
\section{Discussion}
\label{ctr:discussion}
The discussion contains three sections. The first section takes the examples from
\Cref{ctr:theoreticalBackground} into account and how they are expressed
in \ooplss. It is shown how these problems can be modelled in a better
way for some situations and how the programmer is supported by doing this. 
Here, it is considered that the full specification is implemented. The following
section then makes a comparison with Scala an shortly presents how
these problems are treated within Scala and where the major differences
are. The fact that the implementation is not that far as the specification
is discussed in the last section and the status quo of \ooplss'
implementation is presented.

\subsection{Solving prior Problems with \ooplss}
In \Cref{chap:subtypingVsSubclassing} three different examples were
introduced where subtyping forces the programmer to write code that
needs unsafe type casts and runtime checks.

\subsubsection{The Circle - Ellipse Problem}
This example is introduced in \Cref{sec:sharingTypes} where the type
hierarchy differs from the implementation hierarchy. The question is whether an ellipse
is a specialisation of a circle or the other way round.

To consider this example, both definitions of subtyping may be recalled;
In signature subtyping, which is supported by \ooplss, a
definition that a subtype can only add but not remove fields. With this,
an ellipse can be a subtype of a circle since an ellipse has two radii,
i.e., has at least one more field than a circle. In behavioural subtyping
the subtype invariants can only be more restrictive. Here a circle is
a subtype of an ellipse since both radii of an ellipse must be equal
to define a circle.

However, the problem in signature subtyping is that the runtime behaviour
of this model is incorrect, e.g., a circle variable that references a
runtime ellipse object would return a wrong result for the height and
width properties and consequently a wrong area. For this problem
they are different approaches to solve it. One of this is to
decouple both types and implementation completely. Unfortunately this
leads to massively redundant code which of course can be reduced with
various techniques like multiple inheritance and abstract base classes. A
more straightforward solution is now presented in \ooplss in Listing
\ref{lst:ellipseCircleOoplss} using subclasses. Although subsumption
is not not available, the code of \emph{Circle} could be reused and with the
matching relation, new parametrised classes can get introduced that can
handle with both classes properly.

\begin{lstlisting}[float=ht,language=ooplss,caption=Circle-Ellipse problem in \ooplss,label=lst:ellipseCircleOoplss]
class Circle {
	var width : Float;
	def setWidth(widht : Float) : Void { self.width = widht; }
	def area() : Float { return width * width * 3.1415; }
	def width() : Float { return width; }
}

class Ellipse subclassOf Circle {
	def area() : Float { return 3.1415 * width * height; }
	var height : Float;
	def setHeight(r : Float) : Void { self.height = r; }
	def height() : Float { return self.height; }
}
\end{lstlisting}

\subsubsection{Cats and Dogs and Linked Lists}
The example presented in \Cref{sec:recursiveClosure} deals with the
problem of binary methods and consequently with object families.

In the first case binary methods are considered. In a subtyping environment,
binary methods arguments are not changeable with the type hierarchy. This
means that when a programmer wants to implement these binary methods,
unsafe type checks and casts are inevitable. A well known example of this problem is the
equals method as known in Java or \cs. Before two objects can be compared
for equality runtime type checking is necessary although passing other
object types would not make any sense.

The second case is strongly connected with binary methods and the contravariant
restriction for method parameters within subtyping. It deals with the implementation of multiple
object hierarchies with a strong connection between them. There are two
examples that present this very well. First, suppose simultaneously to
an element object hierarchy a hierarchy of operations is defined. For
the \emph{visitor pattern} by Erich Gamma et al. \cite{gamma_design_1995}
is a good solution for a pure subtyping environment and where subsumption
is important. However, if the operations are designed for homogeneous
data structures this has a drawback. If the element hierarchy changes, all
visitors need to be changed which is maybe not possible everywhere. Here
subclassing is maybe more appropriate. \Cref{fig:operationFamily}
presents this idea. With an appropriate parametrisation of the operation
classes it is possible to define appropriate operations for every object.

\begin{figure}[H]
	\centering
	\begin{emp}[classdiag](20, 20)
		Class.AnimalA("Animal")()("mate(Animal): Animal");
		Class.DogA("Dog")()("mate(Dog): Dog");
		Class.CatA("Cat")()("mate(Cat): Cat");

		DogA.ne = AnimalA.s + (-10, -35);
		CatA.nw = AnimalA.s + (10, -35);

		drawObjects(AnimalA, DogA, CatA);
		link(subclassof)(DogA.n -- AnimalA.s);
		link(subclassof)(CatA.n -- AnimalA.s);

		Class.AnimalOp("AnimalOperation[T<#Animal]")()("do(T): T");
		Class.DogOp("DogOperation[T<#Dog]")()("do(T): T");
		Class.CatOp("CatOperation[T<#Dog]")()("do(T): T");

		AnimalOp.w = AnimalA.e + (100, 0);
		DogOp.ne = AnimalOp.s + (-10, -35);
		CatOp.nw = AnimalOp.s + (10, -35);

		drawObjects(AnimalOp, DogOp, CatOp);
		link(subclassof)(DogOp.n -- AnimalOp.s);
		link(subclassof)(CatOp.n -- AnimalOp.s);
	\end{emp}
	\caption{Animal type family with operations on each concrete element}
	\label{fig:operationFamily}
\end{figure}

This idea can easily be adopted to the problem of SinglyLinkedNodes
and DoublyLinkedNodes which is presented in Section
\ref{subsec:matching}. Consequently the node family specifies an element
family where an appropriate list class family does implement appropriate
algorithms for efficient list using.

\subsubsection{Schizophrenic Self References}
The last example considered in this chapter deals with schizophrenic self
references occuring in extension which are introduced in \Cref{sec:schizoReferences}.

This is the most obvious example affected by subclassing and \mytype.

\begin{lstlisting}[float=ht,language=ooplss,caption=Avoid schizophrenic self references with \mytype.,label=lst:solveSchizo]
class Vec2D {
	var x : Int;
	var y : Int;
	def identity : MyType {
		return self;
	}
	def equals(vec : MyType) : Bool {
		return vec.x == self.x && vec.y = self.y;
	}
}

class Vec3D subclassOf Vec2D {
	var z : Int;

	def equals(vec : MyType) : Bool {
		return base.equals(vec) && self.z == vec.z;
	}
}
\end{lstlisting}

\subsection{Status Quo of Implementation within Project}
\label{sec:statusQuo}
\todo{Remember mark}

\section{Further Work}
\label{sec:futureWork}
In projects like this one, years of development could be used for
further improvements and new techniques. Some of these improvements are
discussed here. The section is divided into two fields. One presents some
possible language extensions and the other some implementation details
which would improve a proper use of the language.

\subsection{Language Extensions}
This section presents some possible extensions for the core language
independently of the implementation. The order presented is not weighted
in the sense of its importance. To raise the language to a language
of wide use, all these extensions should be considered as important.

\subsubsection{Pure Object Orientation}
For a straightforward use of numbers and strings with all its available
operators they are implemented as primitive data types. This makes the
implementation easier since the operations does not have to be translated
to method calls. But this enforces special treatment of these primitive types
and the language less extensible since these operators are only available for primitive
types. For a more coherent language the native operators and primitive types should
be dropped for a more sophisticated language where binary and unary operators can be
implemented by arbitrary classes. This would have the advantage that it
would became a pure object oriented language that would not distinguish
between two different types.

\subsubsection{Information Hiding}
Currently, \ooplss does not know any kind of access modifiers like
\emph{public}, \emph{protected} and \emph{private}. Without them no
information hiding is possible which does only provide an object's interface
and not all internal variables and data. For a proper isolation of
internal states and external representation such modifiers should be
introduced.

\subsubsection{Method Overloading and Multiple Extension}
At this time, method overloading is not integrated in \ooplss. Method
overloading would be necessary for certain other concepts, for instance
for multiple inheritance. Introducing method overloading would affect
different constructs which are undesirable for this thesis project since
it is not a core problem which is treated here.

%Currently method overloading is not integrated in \ooplss. With this
%comes several problems which could be avoided. On one hand, it is currently
%not possible to provide multiple inheritance since methods overloading is
%essential for usable class combination. Now, the language is restricted
%that extension is only allowed as long as no method overloading occurs.
%This change can be considered as straightforward since the exact definition
%which method should be called is already possible with full parent
%specification by the caller. 

\subsubsection{Modularisation}
Modularisation is also missing. Today only very monolithic programs
are possible where no separate compilation units are supported. For larger
programs this would increase compilation time and a flexible work between
developer teams is not supported. This makes \ooplss in its current state
useless for serious programs although the target platform is widely
used. Hand in hand supporting multiple files would be inevitable and
necessary for future developments. Even though simple modularisation would be
not very difficult, a more sophisticated form is proposed where modules are
higher-order constructs and can be parametrised \cite{dreyer_type_2003}.

However, with the introduction of proper subclasses a simple form of
modularisation and code reuse is already provided. With this adding
existing code to classes is very simple and safe.

\subsubsection{Interfaces and Abstract Classes}
Additionally to information hiding and modularisation an
enhancement in abstract software definition should be considered. In
\Cref{sec:comparisonScala} it was shown that abstract classes or
interfaces on class level would increase the expressiveness of \ooplss
in several situations. In combination with a powerful module system
the definition of module interfaces or higher-order classes as modules
would come hand in hand.

\subsubsection{Top and Bottom Types}
The language design has a similarity to that of \cpp to be able to not
have an implicit superclass compared to Java's \emph{Object} and \cs's
\emph {Object}. With this it is not possible to write methods that accept
arbitrary objects if no superclass is specified by the programmer and
explicitly extended. This decision is made because two class combinations
are possible and an implicit base class would force the programmer to
use one of them at the very beginning of every class.

Anyway, one proposal to enable the possibility of passing or returning
arbitrary objects would be to introduce a special \emph{Top} type which
contains every possible type. The idea is similar to an \emph{Object}
but which can be used in more cases like value types and reference
types. Actually it is up to the programmer to introduce an own type
which is used as general base type.

%\subsubsection{Higher-Order Functions}

\subsection{Further Implementation Work}
In contrast to \Cref{sec:futureWork}, this section deals with work that
would not change the language itself but would make the language more useful
for daily work.

\subsubsection{Implementing the Specification}
As presented in \Cref{sec:statusQuo} the compiler does not provide the
full language which was specified. This would be the next step in the
implementation that should be done since parametrisation is not supported
yet.

\subsubsection{JDK Integration}
Considering the fact that \ooplss is not fix defined on the current
target platform of the JDK, it would be nevertheless interesting to
improve the integration into the target platform. With this a widely
known and enormous framework would automatically get available to
\ooplss programmers. The problem of this extension would be to integrate
both derivation possibilities which is different to Java. However,
this would may come hand in hand with the direct translation to JDK
Bytecode which is more flexible than the Java language.

\subsubsection{Subclassing with Dynamic Invokes}
The translation from \ooplss to Java is now not very straightforward since
explicit class combinations are necessary. However, in JDK 7 which will
be released in summer 2011 the JDK will get some major changes. One
of this is the introduction of the \emph{dynamicinvoke} command on
Bytecode level. This would support dynamically typed language on the
platform where the most suitable method would be called depending on the
runtime type of the parameter types. Since \ooplss has an own type
checker it would be safe to use dynamic code. With a dynamic method
lookup, the subclassing relation would be much more simple.

\subsubsection{Type Inference}
In many cases the types of a method or variable can be predicted
by the type system with a type inference algorithm. This helps
the programmer to write less code and provide more compact and readable
programs. Without changing \ooplss' semantic, such type inference
could be implemented as a next further step.

\section{Conclusion}
\label{ctr:conclusion}
%estimated pages: 2
