\chapter{Introduction}
Object-oriented programming languages enjoy a wide popularity in
today's developer community. Consequently, different languages are
introduced with different specialities to support the programmer in
daily work. Although they have many different approaches to create
useful, safe and comprehensive code, most of them provide subtyping
polymorphism; which is a well known relation. This makes it fast and
easy to reuse code, but there are some shortcomings in languages that
only provide subtyping as type relation. Since subtyping is restrictive,
it inhibits proper inheritance of binary methods like \emph{equals}. To
overcome this restriction, languages like Java and \cs introduced
type parametrisation, which is better known as \emph{generics} or
\emph{templates} in \cpp. But this still does not solve the problem of
binary methods by extension. Other relations and class combinations are
considered useful in object-oriented languages as well. One of them is
the \emph{matching} relation and the concept of subclassing.

This thesis project aims at designing and developing the
prototypical language \ooplss that integrates both models: on one
hand the model of safe subtyping, on the other hand the model of
safe subclassing. In this language, the programmer decides which
model fits the problem's scope. Generally, exists two names for the
project; OOPLSS\nomenclature{OOPLSS}{Object-Oriented (Programming)
Language with Subtyping and Subclassing} is the project name since it
abbreviates `Object-Oriented (Programming) Language with Subtyping and
Subclassing' and \ooplss is the name of the language and is defined
as a recursive acronym for `{\bf L}ISA {\bf I}ncludes {\bf S}ubcl{\bf
A}ssing'\nomenclature{\ooplss}{{\bf L}ISA {\bf I}ncludes {\bf S}ubcl{\bf
A}ssing}.

This paper only regards statically typed object-oriented languages and does
not deal with dynamic languages at all. The authors assume for this project
that the developing of large software systems benefits from statically
typed languages.

\section{Purpose and Situation}
\subsection{Motivation}
Many programmers consider inheritance as a key feature in object-oriented
languages and as an important characteristic for proper code reuse and
software design. Often, inheritance is put on the same level as subtyping
and is not distinguished. However, many researchers do not agree with
this notion. Instead, they distinguish between two models: subtyping and
inheritance/subclassing \cite{taivalsaari_notion_1996}.
Both models are independent, but not mutually exclusive. Different approaches were
taken to combine both models by using special type variables like \mytype
and type constraints like \emph{exact} types. However, this project aims at
pursuing a different solution by combining both models whereat the programmer has
the full power about what to choose, by introducing two kinds of derivations.
One is called \emph{subclassing} and the other \emph{subtyping}. Depending
on the derivation method, the \mytype variable will get bound at different
times of the derivation process. To test this hypothesis, a prototypical
language was defined and most of these features were implemented.

\subsection{Objectives and Limitations}
Generally speaking, the project consists of two parts. Firstly a language
that contains subclassing and subtyping will be formally defined. The
second part is to implement the language as completely as possible, containing
the lexer and parser, an own type checker and a translation unit to translate
the intermediate representation to Java.

For this project, the special focus lies on the type systems and the
language design. It is not intended to provide a full programming language
that is ready for daily use. Rather, the focus was on showing that the
model works and that some problems, which rise regarding the finding of
design solutions and implementing them, can be solved with this model.

\subsection{Preliminary Activities}
The subject of the project was generally new two both authors. Within
our study program of computer science at the Berne University of Applied
Sciences no major exists on this topic. To gain prior knowledge about
this subject, some introductory courses were taken: \emph{Functional
Programming} and \emph{Automata and Formal Languages}.  One of the
authors has also taken \emph{Computability and Complexity} as a formal
course. He has also written a project paper about type systems with
focus on theoretical backgrounds, practical use and impacts. However,
all of these preliminary works are only generally related with the topics
dealt with in this thesis project. Consequently, this project contains many
topics that were new to the authors, which are listed below.

\subsection{New Learning Contents}
Additionally to the formal learning contents for any bachelor thesis, there were many
subjects which were new to the authors in this specific project. The main new parts are the following:

\squishlist
	\item Design of a programming language from scratch
	\item Designing a compiler from the lexer to the translated output
	\item Implementing a compiler with all its aspects
	\squishlist
		\item Lexing and parsing
		\item Aggregating an abstract syntax tree
		\item Type checking
		\item Translating of intermediate language to concrete output
	\squishend
	\item The parser generator \emph{ANTLR3}\nomenclature{ANTLR}{ANother Tool for Language Recognition, it is a parser generator}
\squishend

\section{Related Works}
Since two different models of class combination are integrated in this
work, several other related papers were considered for this project.
Without these, which were done within the last two decades, this project would 
not have been possible.

The idea for this thesis project was based on the notions from Bruce et. al. as
well as on Abadi and Cardelli. They showed where the
problems in binary methods are, how they can get tackled
and that the notion of subtyping is not sufficient for several
situations \cite{bruce_binary_1995,abadi_subtyping_1996}. These ideas are
based on the work of Canning, who presented the notion of function-bounded
polymorphism in object-oriented languages \cite{canning_f-bounded_1989}.

To adopt a better interpretation of function-bounds
with subtyping, the notion of inheritance and
higher-order subtyping was refined by the following work:
\cite{steffen_higher-order_1994,cook_inheritance_1990,taivalsaari_notion_1996}.
We used these works to understand the correct idea of the models used
in subtyping and subclassing.

A comprehensible presentation of the notion of classification within
both models of subtyping and subclassing was presented by Simons in
\cite{simons_theory_2002-2}; it was his PhD topic. Based on that he
wrote a series of articles within the Journal of Object Technologies.

Within the last fifteen years many other prototypical research
languages were developed to investigate more expressive languages. A
language that contains both subtyping and matching was presented
by Gawecki and Matthes \cite{gawecki_tool:_1995}. Another language
that goes in a similar direction is PolyToil, which is an extension
of Toil with polymorphism invented by Bruce, Schuett and van Gent
\cite{bruce_polytoil:_1995}. The simpler version of PolyToil was published
shortly afterwards and is named $\mathcal{LOOM}$. It's purpose was
to show that subtyping is not as good as intended for object-oriented
languages \cite{bruce_subtyping_1997}. This language is the basis of
the Java extension LOOJ which integrates the ideas of matching and exact
types within Java.

Another aspect of subclassing and subtyping focuses on
flexible software composition without subtyping. One of
the first of these approaches were mixins for a Lisp dialect
\cite{bracha_mixin-based_1990}. This new kind of code reuse was later
refined under the name \emph{traits}, which are behavioural software
units \cite{schaerli_traits:_2003,ducasse_traits:_2006} that are easier
than pure mixins since they only implement behavioural extensions,
not a dedicated state.

\section{Document Structure}
This document is divided into seven chapters, the first being this
introduction. The second chapter on \emph{\nameref{ctr:projectManagement}}
introduces the work plan: how and when what was done and what
methodologies where chosen to reach the target. With the target
following two outputs are meant: this documentation and the
implementation. The chapter is followed by a chapter on the
\emph{\nameref{ctr:theoreticalBackground}}. There the fundamental
backgrounds were outlined. Its purpose is to introduce the reader to
the topic and vocabulary. That chapter also contains three examples
where current languages like Java and \cs reach the limits of their
expressiveness and force the users to bypass the safety of the type
system by using unsafe constructs like \emph{casts}. The following
chapter on \emph{\nameref{ctr:languageSpecification}} contains
the main specification of the language \ooplss. After the first
definitions the language is introduced on an incremental way where
the semantics of the used constructs are presented. Then the chapter
on \emph{\nameref{ctr:swDesignImplementation}} shows how the software
is structured and how the compiler is designed. This is followed by
the chapter on \emph{\nameref{sec:comparisonScala}} which compares
\ooplss to some structures of Scala\nomenclature{Scala}{Abbreviates
\emph{A Scalable Language} and is an advanced multi-paradigm
programming language designed at the EPFL}. A comparison shows how the
\emph{self}-variable is explicitly, respectively implicitly typed and
another one takes regard on how software can be composed in Scala and
\ooplss with traits and subclassing respectively.  The last chapter is
\emph{\nameref{ctr:discussionConclusion}} and discusses the examples
introduced in chapter \emph{\nameref{ctr:theoreticalBackground}} and how
those can be expressed with \ooplss. Some further work on the language
is proposed as well.

After the regular content follows the appendix, containing a manual, the type
inference rules, some code examples, the work report and the original
definition of the project.

