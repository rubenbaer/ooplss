%**********************************************************************************
%
% o	 o	 o	 o					Berne University of Applied Sciences
%						 :					Engineering and Information Technology
%						 :......o	 Computer Science Division
%
% OOPLSS - Object-Oriented Language with Subtyping and Subclassing
% Bär Ruben, Heinemann Stefan
%**********************************************************************************
\subsection*{Abstract}\todo{Early draft version}
Subtyping is a well known type relations in object-oriented
programming languages and is an important aspect of these languages
to support sound type polymorphism. To support type polymorphism,
subtyping has to restrict the subtype declaration to allow safe type
substitutions. This prohibits the specialisation of binary method declarations 
in subtypes. In many popular object-oriented programming languages, the type
hierarchy is implicitly the same as the implementation hierarchy. This
contradicts the notion of inheritance as code reuse and subtyping as type
substitution. To solve this type of problem, parametrisation has been introduced in
different languages which is also known as generics. But these do not
tackle recursive types used for binary methods. To solve this problem
the concept of subclassing with a special type variable \mytype has been
introduced which is an implicit type parameter. But languages supporting
the \mytype variable do not allow the programmer to choose the correct
binding of the variable that fits most to the problem under consideration.

Specialising binary methods without bypassing the type system with
unsafe constructs like casts is often not possible since subtyping is
too restrictive even though it is not required for every problem. This
rises the question whether it is possible to introduce a type safe method
for specialising binary methods or not.

This thesis project presents a solution that allows the programmer to
choose the accurate model that fits the problem. This is presented
with the prototypical programming languages \ooplss. To test the
assumptions, an implementation of some of the features was created
within this project. \ooplss supports two kinds of class derivation
with different specialisation semantics. New to subtyping and generics
is a proper subclassing model which supports inheritance.

It could be shown that subclassing and subtyping is not a contrary model
that can fit in one language. It avoids complex structures for typing
binary methods and includes the possibility of safe software composition
with less restrictions than subtyping.

%What is the problem or question that the work addresses?
%Why is it important?
%How was the investigation undertaken?
%What was found and what does it mean?


%\vfill
%%\subsection*{Zusammenfassung}
%\selectlanguage{ngerman}
%Hier kommt der deutsche Abstract
%
%\selectlanguage{british}
%\vfill
