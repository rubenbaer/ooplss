\nomenclature{Subtyping}{A relation between two type expressions with safe substitutability.}
\nomenclature{Subclassing}{A class combination that does not obey boundaries of a subtype relation.}
\nomenclature{Class Combination}{Refers to both, subtyping and subclassing. A class combination aims to code reuse.}
\nomenclature{Binary Method}{A method of an object that expects an object of the same type.}
\nomenclature{Subsumption}{}
\nomenclature{Nominal Subtyping}{}
\nomenclature{Structural Subtyping}{}
\nomenclature{Matching}{}
\printnomenclature

\appendix
\part*{Appendix}

\chapter{Manual}
% estimated pages: 3

%\chapter{Tool Evaluations}
%estimated pages: 3

\chapter{Type Rules}
\label{ctr:typeRules}
\begin{itemize}
\item Product Introduction
\item Product Elimination
\item Function Introduction
\item Function Elimination
\item Record Introduction
\item Record Elimination
\item Record Subtyping
\item Function Subtyping
\item Product Subtyping
\item Record Extension
\item Record Overriding
\item Recursive Subtype
\item Subclass
\item Classify
\end{itemize}

\chapter{Example Codes}
\label{ctr:exampleCode}

%\chapter{Syntax Summary}

\chapter{Journal}
\begin{center}
\begin{longtable}{|c|p{12cm}|}

\hline \multicolumn{1}{|c|}{\textbf{Week}} & \multicolumn{1}{c|}{\textbf{Journal Entry}} \\ \hline 
\endfirsthead

\hline \multicolumn{1}{|c|}{\textbf{Week}} &
\multicolumn{1}{c|}{\textbf{Journal Entry}} \\
\endhead

\hline \multicolumn{2}{|r|}{{Continued on next page}} \\ \hline
\caption[Journal]{Journal} \label{table:journal} \\
\endfoot

\multicolumn{2}{c}%
{{\bfseries \tablename\ \thetable{} -- continued from previous page}} \\
\endlastfoot

1 & 
\begin{journal}
	\item Project structure defined
	\item Repository installed
	\item Major task identified
\end{journal}
\\ \hline
2 & 
\begin{journal}
	\item Documented the theoretical background
\end{journal}
\\ \hline
3 & 
\begin{journal}
	\item Documented the theoretical background
\end{journal}
\\ \hline
4 & 
\begin{journal}
	\item Documented the theoretical background
	\item Defined a basic draft of the language's syntax
	\item Decision between JavaCC and ANTLR -> ANTLR chosen
\end{journal}
\\ \hline
5 & 
\begin{journal}
	\item Started the ANTLR grammar
	\item Writing gUnit tests for the grammar
\end{journal}
\\ \hline
6 & 
\begin{journal}
	\item Extended the ANTLR grammar
	\item Started to write AST rewrite rules
	\item First meeting with Mr. Spichiger
\end{journal}
\\ \hline
7 & 
\begin{journal}
	\item Reorganised the second part of the document
	\item A more theoretical definition of subtyping introduced
\end{journal}
\\ \hline
8 & 
\begin{journal}
	\item Changed repository from subversion to git
	\item 
\end{journal}
\\ \hline
9 & 
\begin{journal}
	\item Resolved the abiguity of the extension mechanism
	\item Partial definition of the code translation from \ooplss to Java
\end{journal}
\\ \hline
10 & 
\begin{journal}
	\item Restructured the language speficification part in a syntax and semantis part
	\item Noted the typing rules that are needed (Not yet written)
	\item StringTemplate enginge selected for code generation
\end{journal}
\\ \hline
11 & 
\begin{journal}
	\item Language specification further written
	\item Ant package builder added
	\item Resolved problem with \self-variable and \mytype
	\item Subtyping relation added
\end{journal}
\\ \hline
12 & 
\begin{journal}
	\item 
\end{journal}
\\ \hline
13 & 
\begin{journal}
	\item Second expert meeting
\end{journal}
\\ \hline
14 & 
\begin{journal}
	\item 
\end{journal}
\\ \hline
15 & 
\begin{journal}
	\item 
\end{journal}
\\ \hline
16 & 
\begin{journal}
	\item 
\end{journal}
\\ \hline
\end{longtable}
\end{center}


\chapter{Definition of Project}
\begin{figure}[H]
\centering
%\includegraphics[height=\textheight,keepaspectratio=true]{../proposal}
\setlength\fboxsep{0pt}
\setlength\fboxrule{0.5pt}
\fbox{
\includegraphics[scale=0.58]{../misc/proposal}
}
\end{figure}
