\nomenclature{Subtyping}{A relation between two type expressions with safe substitutability.}
\nomenclature{Subclassing}{A class combination that does not obey boundaries of a subtype relation.}
\nomenclature{Class Combination}{Refers to both, subtyping and subclassing. A class combination aims to code reuse.}
\nomenclature{Binary Method}{A method of an object that expects an object of the same type.}
\nomenclature{Subsumption}{Characterises a expression that can be of type \A as well as of type \B if \A is a subtype of \B.}
\nomenclature{Nominal Subtyping}{The subtype relation is explicitly specified by the programmer with a language keyword.}
\nomenclature{Structural Subtyping}{The subtype relation is given implicitly between to types. This is more flexible then nominal subtyping but more difficult in use and type checking.}
\nomenclature{Matching}{A type relation that is the same as subtyping with absence of \mytype but differs in the present of \mytype. Matching does not have the subsumption property.}

\clrpage
\printnomenclature
