\chapter{Deployment}

\section{Building}

To build the project, there is an Apache ANT\footnote{\href{http://ant.apache.org/}{http://ant.apache.org/}}
build file. It contains several targets to build the project.
The most important are the following: 

\begin{description}
\item[build] Build the project. This is the default target. It runs
the tests, generates the ANTLR parsers and then builds the source
code. The result is a jar file called \emph{ooplss.jar}.
\item[clean] Cleans the whole project by removing files that are not
actually part of the project but created during the build process.
\item[doc] Builds the documentations. This includes on one hand all
the PDFs from the \LaTeX files and on the other hand the code
documentation with Doxygen.
\item[latex] Similar to the \emph{doc} target but only builds the
latex files.
\item[dist] Builds the whole project and then moves the
according file to a \emph{dist} directory, which then
includes all files needed for running the compiler and the documentations.
\item[pack] Creates the distribution and then a tgz and a zip archive containing
the whole project including the distribution folder.
\item[run-tests] Run all the unit tests (gUnit and JUnit).
\end{description}

To run one of these target simply following line has to be entered:
\begin{lstlisting}[language=bash,caption=Running the ant build system]
ant [target-name]
\end{lstlisting}

\section{Installation}

Another target of the ANT build system is the \emph{install} target.
This target only works on a Unix-System. It needs to be run with root
privileges.

This target copies various files to the system:
\begin{itemize}
\item the compiler executable to \emph{/usr/share}
\item the libraries to \emph{/usr/share/ooplss}
\item the manpage to \emph{/usr/share/man/man1}
\end{itemize}

\section{Software Dependencies}

\subsection{Building}
Following software needs to be installed on the system where the 
build is made:

\begin{itemize}
\item The Java Compiler (Version 1.6)
\item The texlive-latex distribution
\item Apache ANT
\item Metapost
\item Make
\end{itemize}

The ANTLR3 package comes with the project since we needed a specific version
to be able to run the gUnit tests correctly.


\subsubsection{Microsoft Windows}

For the building, some common Unix-Tools are required. In case of a 
Microsoft Windows operating system, they need to be installed extra, for
example with the GnuWin32 package\footnote{\href{http://gnuwin32.sourceforge.net/}{http://gnuwin32.sourceforge.net/}}.

\subsection{Execution}

For the execution of the compiler, only Java version 1.6 is required.

\subsection{Eclipse Integration}

The project can easily be integrated into Eclipse, since the Eclipse
configuration file (\emph{.project}) is also present in the root of the
project. Like this, all the necessary build paths are set and also the
build targets are integrated automatically.

\section{Distribution Structure}

\begin{description}
\item[examples/] Some example source codes written in ooplss
\item[html/] Source code documentation generated with doxygen
\item[lib/] All necessary libraries for ooplss.jar
\item[manpage.pdf] Manual for ooplss and ooplss.cmd converted to PDF
\item[ooplss] Startup bash script for the compiler
\item[ooplss.1] Manual page source
\item[ooplss-book.pdf] Documentation for a two sided print output
\item[ooplss.cmd] Startup cmd script for the compiler
\item[ooplss.jar] Java archive containing the compiler
\item[ooplss-print.pdf] Documentation for a two sided print output
\item[ooplss-web.pdf] Documentation optimised to read online
\item[README] Project description
\item[source.pdf] Source code documentation generated with doxygen
\end{description}

