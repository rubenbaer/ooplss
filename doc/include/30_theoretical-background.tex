\chapter{Theoretical Background}
\label{ctr:theoreticalBackground}
In modern object-oriented programming languages \emph{subtyping}
is the fundamental and only concept used for class
combination. \nomenclature{Class Combination}{Refers to both, subtyping
and subclassing. A class combination aims to code reuse} Often yet
imprecisely it is called \emph{subclassing} as well. This chapter
investigates the differences between the two concepts of subtyping and
subclassing. Even though subtyping has its advantages it is also
limited in expressiveness when binary methods are used, which appear in
a programmer's daily work.

This part introduces these two concepts, indicates where subtyping and
where subclassing is preferred and explains where the difference are.
Generally, subtyping is used for polymorphism in object-oriented
languages, whereas subclassing is not. However, subclassing can be
used for code reuse where subtyping is inappropriate since subclassing
enables, contrary to subtyping, further specialisation of methods.
To make subclassing practical, an additional relation needs to be
introduced: \emph{matching}. Type quantification with the matching
relation extends the language with function-bounded subtyping or with
higher-order subtyping, depending on which interpretation of matching
is used. The benefit of using matching for type quantification is that
it takes special regard on recursive type definitions.

\section{Rules of Inference}
\label{sec:rulesOfInference}
In this chapter and in the appendix rules of inference are used
to specify type declarations. Generally, a rule of inferences consists
of $\mathbb{N}_0$ premisses and a conclusion written as follows:

\begin{mathpar}
  \inferrule*[right=Rule-Name]
    {Premise_1 \\ Premise_2 \\ \ldots \\ Premise_n}
    {Conclusion}
\end{mathpar}

Whenever the premisses can be logically derived, it can be concluded
that the conclusion is satisfied. If a rule contains no premises, then the rule
is used as axiom which is always true. As a concrete example of the syntax the
modus ponens is written in rule notation as follows, where $\Rightarrow$
denotes an implication:

\begin{mathpar}
  \inferrule*[right=Modus-Ponens]{A \Rightarrow B \\ A}{B}
\end{mathpar}

Another often used notation is the type annotation $x:T$ which denotes
that $x$ is of type $T$. It can also be understood as set operator
$x \in T$, which is not quite precise though.

\section{Nominal and Structural Type Systems}
\nomenclature{Nominal Type System}{Nominal or nominative type system are
name based type systems. The relation between two types is explicitly
specified with the use of keywords. In \ooplss \emph{subtypeOf} and
\emph{subclassOf}}

\nomenclature{Structural Type System}{Structural type systems are the
counterpart of nominal type systems. Type relations are implicitly
specified and determined by the type system it self. Nevertheless both
can be combined within one model}

Generally, two different kinds of type systems can be distinguished,
depending on how type relations are determined. The first and most
flexible technique is provided by a structural type system which is
used in more sophisticated languages and research topics. A structural
type system does not take the object combination operators which are
used into account, but rather takes care of the object's structure in
order to determine the relations between two objects. The other family
of type systems is of nominal nature which is used in popular languages
like Java, \cs, \cpp and many more. For nominal type systems the type
relation between objects is specified with keywords that are used for
class combination. For example, Java provides an `\emph{extends}' keyword
which makes the combination and restricts the programmer to define only
classes that are in subtype relation to its parent. This restriction is
not available in structural type systems which are more complicated in
type checking, whereas in nominal type systems the relation is specified
explicitly \cite{malayeri_integrating_2008,pierce_types_2002}.

In current research some approaches are being performed aiming
at providing the programmer with both possibilities, see
\cite{findler_semantic_2004,gil_whiteoak:_2008,malayeri_integrating_2008}.

\section{Subtyping and Subclassing}
\label{chap:subtypingVsSubclassing}
In this chapter we want the reader to become familiar with the terms
\emph{subtyping} and \emph{subclassing} and their differences. Both
concepts exist in some of the widely known object oriented programming
languages like Java, Eiffel or \cpp, but are usually mixed up in the
used terminology. For proper distinction, the following terminology
is used: \emph{extension}\nomenclature{Extension}{The operation
used for subtyping as known in Java} is used for subtyping and
\emph{inheritance} \nomenclature{Inheritance}{The operation for
subclassing, i.e. combining classes without creating subtypes} for
subclassing. The term \emph{derivation}\nomenclature{Derivation}{Refers to
both, extension and inheritance. A class combination aims to code reuse}
is used to refer to both concepts without referencing to one in special.

\subsection{Subtyping}
\nomenclature{Subtyping}{A relation between two type expressions with
safe substitutability} Subtyping is a binary relation, denoted with
`<:', which defines a substitution relation between two types and is one
style of type polymorphism. Assuming that a type $B$ is a subtype of $A$,
subtyping then ensures that $B$ has compatible methods and data fields,
together simply called \emph{fields}\nomenclature{Field}{A field
within an object and can be both, a method or a data
field representing the object's state}, to its \emph{supertype} $A$. This
implies that wherever type $A$ is expected, type $B$ can be used.

To ensure type compatibility between $A$ and $B$ subtyping has the
following properties:

\begin{defn}[Subtype reflexivity]
	\label{def:subtypeReflexivity}
	\begin{mathpar}
		\inferrule*{\\}{A <: A}
	\end{mathpar}
\end{defn}
\begin{defn}[Subtype transitivity]
	\label{def:subtypeTransitivity}
	\begin{mathpar}
		\inferrule*{B <: A\\C <: B}{C <: A}
	\end{mathpar}
\end{defn}

Type safe subtyping is defined for objects and fields. Subtyping for
objects is defined as follows: If $B$ <: $A$ then $B$ needs at least as
many fields as $A$ entails; the former is in subtype relation to the
latter one. Furthermore, $B$ can contain other fields which
do not have any correspondence in $A$. Since objects are often considered
as records\nomenclature{Record}{A record is a tuple data structure with
named fields} -- which are ordered -- in subtyping, permutations of those
are defined as equal.
\begin{defn}[Object subtyping]
	\label{def:subtypeObject}
	If $B := \{l_b : Y_b\}$, $A := \{l_a : X_a\}$, $B <: A$, $A$
	contains $a \in \mathbb{N}$ fields, $B$ contains $b \geq a$
	fields then for each $i\leq a$ it is true that $Y_i <: X_i$.
\end{defn}

For fields, the subtyping relation differs for methods and data fields.

Data fields need to be type invariant in a subtyping context since
subtyping is antisymmetric, i.e., $B$ <: $A$ and $A \neq B$ imply $A \nless: B$.

Subtyping method types, denoted as $A \rightarrow B$ for a function with
domain $A$ and codomain $B$, is a bit more complicated since parameters
and return values have to be considered separately. On one hand the return
type can change covariantly with the type hierarchy and on the other hand
the parameters are only allowed to change in a contravariant way, i.e.,
change upwards in the hierarchy.

\begin{defn}[Method subtyping]
\label{def:methodSubtyping}
	\begin{mathpar}
		\inferrule*{X <: A\\B <: Y}{A \rightarrow B <: X \rightarrow Y}
	\end{mathpar}
\end{defn}

\subsubsection{Variance}
Variance classifies the type compatibility within a type hierarchy. The
different types of variance were are introduced in category theory and
were adopted to type theory \cite{pierce_basic_1991}. In a subtyping
environment, method parameter types and method return types can be
contravariant and covariant, respectively. State variables can only be
redefined invariantly.

\begin{description}
	\item[Invariance] No type change is allowed.
	\item[Covariance] Covariance denotes that type changes are only
	allowed downwards within the hierarchy. For example,
	it is assumed that \B <: \A and \Y <: \X and \T is an arbitrary
	type. Now, \A has a method $m$ of type $T \rightarrow X$. If
	\B replaces $m$ with $m^\prime$ the return type of $m^\prime$
	has to be covariant. Consequently the type of $m^\prime$ can be
	$T \rightarrow Y$ since the return type is covariant to \B <:
	\A. The opposite direction where $m^\prime$ replaces $m$ in a
	subtype is not allowed since the return type would change from
	a more specific to a more general one which is contravariant
	and is not allowed within subtyping.
	\begin{figure}[H]
		\centering
		\begin{emp}[classdiag](20, 20)
			Class.A("A")()("+m(arg: T):X");
			Class.B("B")()("+m(arg: T):Y");

			B.e = A.w + (-20, 0);
			drawObjects(A, B);
			link(inheritance)(B.e -- A.w);
		\end{emp}
		\caption{Correct covariant method return types.}
		\label{fig:covariance}
	\end{figure}

	\item[Contravariance] Contravariance is the opposite of
	covariance. Type changes are only allowed upwards in the
	type hierarchy. The types of function parameters have to be
	contravariant. For example, if \B <: \A, \Y <: \X, \T is
	an arbitrary type and $A$ contains a method $m$ of type $X
	\rightarrow T$ which is replaced in \B by method $m^\prime$,
	then $m^\prime$ can be of type $Y \rightarrow T$ since the
	parameter type is contravariant to \B <: \A.
	\begin{figure}[H]
		\centering
		\begin{emp}[classdiag](20, 20)

			Class.AA("A")()("+m(arg: Y):T");
			Class.BB("B")()("+m(arg: X):T");

			BB.e = AA.w + (-20, 0);
			drawObjects(AA, BB);
			link(inheritance)(BB.e -- AA.w);

		\end{emp}
		\caption{Correct contravariant method parameter types.}
		\label{fig:contravariance}
	\end{figure}
\end{description}

\subsubsection{Liskov Substitution Principle}
\begin{quotation}
`Let $q(x)$ be a property provable about objects $x$ of type $T$. Then
$q(y)$ should be true for objects $y$ of type $S$ where $S$ is a subtype
of $T$.' -- Barbara Liskov \cite{liskov_behavioral_2001}
\end{quotation}

With this Liskov Substitution Principle, short LSP, Barbara Liskov
provided a formal and strong description of subtyping. This
notion of subtyping implies behavioural subtyping. In behavioural
subtyping the correctness of the given properties has to be
provable. Thus, since behavioural subtyping requires a proof system
which is generally undecidable, the whole system becomes undecidable
\cite{poll_behavioural_1998}. However, in languages without such a strong
property, e.g., languages with signature subtyping, the type system can
become decidable. Nevertheless, the type systems with signature subtyping
can not proof the correctness of the program. In signature subtyping
the semantics of a program is not considered by the type system, only
the object and field signatures are important and being checked. In the
following text, subtyping does always referring to signature subtyping.

LSP introduces a subsumption rule for type declarations though:
\nomenclature{Subsumption}{Characterises a expression that can be of
type \A as well as of type \B if \A is a subtype of $B$}

\begin{defn}[Subsumption]
\label{def:subsumption}
	\begin{mathpar}
		\inferrule*{x : A\\A <: B}{x : B}
	\end{mathpar}
\end{defn}

\subsection{Types Versus Implementation}
\label{sec:sharingTypes}
Although the terminology of \emph{subtyping}, \emph{extension} and
\emph{inheritance} is used mixed up in the context of languages like Java
or \cpp, the concept in use in these languages is actually subtyping and not subclassing.
The word \emph{type} in type theory is similar to sets in set theory. A
type system usually contains two kinds of types. Primitive types,
e.g., integers, booleans in Java or \cpp or object types defined with
classes in class based object-oriented languages. They determine the
possible states of an object and what operations can be performed on them
or rather which messages they can receive.

In object-oriented languages like Java and \cpp, the type hierarchy
is directly linked to the implementation hierarchy, in other words,
derivation automatically creates a subtype and there is no escape from
this. From a practical point of view however, derivation is merely a
mechanism to reuse code in further specialisations of types. This leads
to the first conflict of types and implementation. This problem of
the mixed hierarchy is illustrated in \Cref{fig:implementationConflict}
\cite{simons_theory_2003-4}.

\begin{figure}[H]
	\centering
	\begin{emp}[classdiag](20, 20)
		Class.Shape("Shapes")()("+move(x, y):Integer", "+drawOn(c: Canvas)");
		Class.Ellipse("Ellipse")("f1, f2: Point", "r, s: Integer")();
		Class.Rectangle("Rectangle")("o: Point", "w, h: Integer")();
		Class.Circle("Circle")("{f1 = f2, r = s}")();
		Class.Square("Square")("{w = h}")();

		Ellipse.ne = Shape.s + (-10, -20);
		Rectangle.nw = Shape.s + (10, -20);
		Circle.n = Ellipse.s + (0, -20);
		Square.n = Rectangle.s + (0, -20);

		drawObjects(Shape, Ellipse, Rectangle, Circle, Square);
		link(inheritance)(Ellipse.n -- Shape.s);
		link(inheritance)(Rectangle.n -- Shape.s);
		link(inheritance)(Circle.n -- Ellipse.s);
		link(inheritance)(Square.n -- Rectangle.s);

		Class.Point("Point")("x, y: Integer")();

		Point.w = Shape.e + (100, 0);
		Class.CircleR("Circle")("r: Integer")();
		Class.SquareR("Square")("w: Integer")();
		Class.EllipseR("Ellipse")("p, q: Integer", "s: Integer")();
		Class.RectangleR("Rectangle")("h: Integer")();

		CircleR.ne = Point.s + (-10, -20);
		EllipseR.n = CircleR.s + (0, -20);
		SquareR.nw = Point.s + (10, -20);
		RectangleR.n = SquareR.s + (0, -20);

		drawObjects(Point, CircleR, SquareR, EllipseR, RectangleR);
		link(inheritance)(CircleR.n -- Point.s);
		link(inheritance)(EllipseR.n -- CircleR.s);
		link(inheritance)(SquareR.n -- Point.s);
		link(inheritance)(RectangleR.n -- SquareR.s);
	\end{emp}
	\caption{Sharing types vs implementation.}
	\label{fig:implementationConflict}
\end{figure}

The hierarchy on the left side expresses the conceptual family; a
\emph{Circle} is a special kind of \emph{Ellipse}, so is \emph{Square}
of \emph{Rectangle}. These objects do not actually `extend' their
parents though, instead, they rather remove properties. For instance,
the \emph{Circle} includes the property of having two equal radii. The hierarchy
on the right is conceptual nonsense. From the implementation point of
view however, it shows how one could want to derive and reuse code, in
this case the x and y coordinates. Therefore, inheritance (and reuse) of
code on one hand and subtyping on the other should be treated differently.

\subsection{Binary Methods}
\label{sec:recursiveClosure}
\nomenclature{Binary Method}{A method of an object that expects an
object of the same type} Another problem of the subtyping concept in
object-oriented languages is the contravariance of method argument as
described in \Cref{sec:rulesOfInference}.

This restriction perfectly makes sense for the type safety in programming
languages. For certain class constellations it leads to further problems
though.

Assuming an object \emph{Animal} that has the method \emph{mate}
taking an argument of the type \emph{Animal} and returning a type
\emph{Animal}. Intuitively, all other animal types, such as \emph{Dog}
or \emph{Cat} would be derived from this type. Yet they would want
to have more specific arguments that only accept values of its own
type, so that dogs can only mate with dogs and cats only with cats
\cite{simons_theory_2003-1}. Figure \ref{fig:animalContravariance}
illustrates this situation. However, because of the contravariance
restriction in subtyping this cannot be achieved by overriding the method
with a more specific argument type.

\begin{figure}[H]
	\centering
	\begin{emp}[classdiag](20, 20)

		Class.Animal("Animal")()("mate(Animal): Animal");
		Class.Dog("Dog")()("mate(Dog): Dog");
		Class.Cat("Cat")()("mate(Cat): Cat");

		Dog.ne = Animal.s + (-10, -25);
		Cat.nw = Animal.s + (10, -25);

		drawObjects(Animal, Dog, Cat);
		link(inheritance)(Dog.n -- Animal.s);
		link(inheritance)(Cat.n -- Animal.s);

	\end{emp}
	\caption[More specific arguments.]{The arguments are more specific, which is not allowed because of the contravariance rule.}
	\label{fig:animalContravariance}
\end{figure}

A similar problem arises in the method \emph{Object.equals: Object
$\rightarrow$ Boolean} in Java. The class \emph{Object} has an
implementation of the equals method that takes an argument of the type
\emph{Object}. All objects automatically are of the type \emph{Object}
in Java. Because of the constraint of contravariance, it is not allowed
to override this method with a more specific argument type which however
would be reasonable since it does not make sense to compare two objects
of different types. Instead, if a programmer wants to implement a new
version of this method, the use of type casts is inevitable.

To overcome this, a language would need to specialise parameters in a
covariant way. This is possible in Eiffel but it makes the type system
unsafe. Later on, a type safe alternative will be presented.

\subsection{Schizophrenic Self-Reference}
\label{sec:schizoReferences}
In objects there has to be some recursion variable that refers to the
object itself, in order to be able to call methods or use values on
self. In some languages this is called \self or \emph{this}, in \ooplss
the term \self will be used to refer to this variable. Two main cases
where recursion occurs can be identified \cite{simons_theory_2003-2}.

\begin{enumerate}
	\item When a method accesses methods or properties of the self object
	\item When a method has arguments or return values of the same type as the object
\end{enumerate}

The first case is problematic when objects are derived; it is not
quite clear to which object the self variable should refer after
the derivation. In \cpp and Java, the self-reference in extended
methods refers to the base object that was derived. This means that
an object may contain many versions of self-references, which is
called \emph{schizophrenic self-reference}. Other languages like
Smalltalk and Eiffel treat the self-reference differently; they
redirect the self-reference to the derived object. Consequently
both models have different semantics in recursion which
are called \emph{naive recursion} and \emph{mutual recursion}
respectively \cite{cook_denotational_1989}. Both are illustrated in
\Cref{fig:schizoRecursion} where $M$ derives $F$ which is a recursive
function. In the naive case the recursive call of the parent method, is
not changed which is different to mutual recursion where the recursive
call in $F$ is returned to the caller $M$ \cite{simons_theory_2003-2}. In
\cpp there is the possibility to declare a method virtual to achieve
mutual recursion. In Java, all methods are implicitly virtual.

\begin{figure}[H]
	\centering
	\subfloat[Naive recursion]{
		\includedot[scale=0.7]{dot/naiveRecursion}
	}
	\subfloat[Mutual recursion]{
		\includedot[scale=0.7]{dot/mutualRecursion}
	}
	\caption[Naive and mutual recursion where M <: F.]{Naive and mutual recursion where M <: F \cite{cook_denotational_1989}.}
	\label{fig:schizoRecursion}
\end{figure}

An example of schizophrenic self-reference of the second case is a two
dimensional vector type \emph{Vec2D} which has two coordinates and the
methods \emph{identity} and \emph{equals}. The identity method returns
the reference to the object itself and the equals method compares
two objects of this type. It therefore takes an argument of the type
\emph{Vec2D}. Let's assume there is a subtype \emph{Vec3D} that derives all
these methods, adds a z-coordinate and overrides the equals method. The
equals method of \emph{Vec3D} then clearly takes an argument of the type
\emph{Vec3D}. Listing \ref{fig:schizoListing} shows this constellation
in Java \cite{simons_theory_2003-2}.

\begin{lstlisting}[float=ht,caption={An example of schizophrenic self-reference.},label={fig:schizoListing}]
class Vec2D {
	int x,y;

	Vec2D identity() {
		return this; // A Vec2D self variable
	}

	boolean equals(Vec2D vec) {
		return vec.x == this.x && vec.y == this.y;
	}
}

class Vec3D extends Vec2D {
	int z;

	boolean equals(Vec2D vec) {
		// Self is of Vec3D type
		if (vec instanceof Vec3D)
			// Uses insecure type casting
			return super.equals(vec) && this.z == ((Vec3D)vec).z;

		return false;
	}
}
\end{lstlisting}

In the derived class \emph{Vec3D} the self reference in the equals
method refers to a \emph{Vec3D} object while the self-reference in the identity method
refers to a \emph{Vec2D} object. This is why an instance of the class
\emph{Vec3D} would be schizophrenic, i.e., containing different versions
of \emph{self}.

This is how the type of self-references is bound in subtyping; before
the class combination or extension, respectively, of objects is done.
In the concept of subclassing however, it is done after the combination.

\subsection{Subclassing}
\nomenclature{Subclassing}{A class combination that does not
obey boundaries of a subtype relation}

The goal of the concept of subclassing is now to provide a different kind
of derivation which is called \emph{inheritance}, with what the programmer
should be able to inherit code without automatically creating a subtype
\cite{simons_theory_2002-2}. This tackles the problem described in
\Cref{sec:sharingTypes}. Since subclassing does not create subtypes, the
restriction of contravariance, described in \Cref{sec:recursiveClosure},
is not relevant anymore. And as already mentioned, the fixation of the
self references and types is done after class combination, so there
will be no schizophrenic self-references.\footnote{See first case in
\Cref{sec:schizoReferences}} Now, since subclassing is less restrictive
than subtyping, a class that is a subclass of another one
is also less limited in its behaviour and
can exceed the behavioural expectation of the superclass which is not
the case for subtyping \cite{simons_theory_2002-2}.

One possibility of subclassing is extending subtyping with type
parametrisation which enables the possibility of contravariant
method types since they can be specified later, after the class
definition. However, the subtype relation within parametrised types
can not define a transitive closure as long as a type parameter is not
annotated whether the variable can be used in method parameter type
declarations or as method return type declaration. As long as the type
variable is allowed to appear in every type declaration, the subtyping
relation can not be enforced between between two objects with different
parameters. Java introduced wildcards in type parameters and \cs
allows explicitly to declare input and output type parameters to allow
subtyping polymorphism of parametrised types. These solutions still
suffers from one problem. How to type recursive types where the fields
have to be compatible with the \emph{self}-reference.\footnote{See second
case in \Cref{sec:schizoReferences}} Here, the problem is that passing
the own type as type parameter needs a recursive type declaration which
can not be declared by hand.

To tackle this issue, the type parametrisation can be performed
implicitly by introducing the generic type variable \emph{MyType}. Since
it is possible to have infinite recursive types using the fixed point
theorem \cite{pierce_types_2002} it is possible to declare recursive
types implicitly. In subclassing the \mytype variable is bound after
the class combination similar to normal non-recursive type parameters
which is different in a subtyping environment.

\section{Making Subclassing Practical}
\label{ctr:makingSubclassingPractical}
% estimated pages: 2-3
One of the most fundamental and well known relations between recursive
types in today's programming languages is the subtype relation.
In languages where subclasses do not implicitly define subtypes a
more general relation is needed -- \emph{matching} -- which is weaker
and does not support subsumption. Originally, \emph{matching} was
proposed as function-bounded subtyping\nomenclature{Function-Bounded
Subtyping}{Function-bounded subtyping is a bounded type quantification
where an expression is polymorph to subtypes of the bounded type
constrained} \cite{canning_f-bounded_1989}. Function-bounded
subtyping extends simple subtyping with parametrisation which
is known as \emph{generics} in well known programming languages
like Java or \cs \cite{barron-estrada_inheritance_2003}. Another
interpretation of \emph{matching} as higher-order subtyping with
better properties such as reflexivity and transitivity was shown in
\cite{abadi_subtyping_1996}. This interpretation is more difficult for
the users of such a language since the semantics is complex and not easy
to understand in all circumstances because subtyping is on the object
generator level. However, this is specified for structural but not for
nominal type systems.

\subsection{Matching}
\label{subsec:matching}
\nomenclature{Matching}{A type relation that is the same as subtyping with absence of \mytype but differs in the present of \mytype. Matching does not have the subsumption property}
\begin{prop}[Matching]
	\label{prop:matching}
	Bruce defines matching as:
	\quote{`the same as subtyping in the absence
	of the \emph{MyType} construct, but differs in the presence of
	\emph{MyType}, because \emph{MyType} implicitly has different
	meanings in different types.' -- \cite{bruce_foundations_2002}}
\end{prop}

A solution proposed by Bruce \cite{bruce_binary_1995} is
\emph{matching}. Matching is a generalised subtyping relation between to
types, written as `\match'. It takes a \mytype in subclasses into account.
An important difference is that the matching relation is only defined
among object types and not among field types, like it is in subtyping. As
said by Bruce in \cite{bruce_foundations_2002} matching does not differ
very much from subtyping except that \mytype is not substituted by its
concrete object type. This makes the relation weaker than subtyping
where contravariant method parameters are not allowed. Depending on the
interpretation of matching it comes to different definitions. Since
only the function-bounded interpretation is relevant for this project, this
definition is given in Definition \ref{def:matchingAsBound}.

Listing \ref{lst:javaBinary} contains some problems that can
not be tackled with subtyping. First of all the variable in line
\ref{lst:javaBinary0} can not be specialised since it is invariant. This
entails that even when it is allowed to specialise the method \emph{getNext}
in the subclass -- which is not the case in Java -- a cast operation
is necessary to get the correct type in the overridden method on line
\ref{lst:javaBinary1}.

The second problem is on line \ref{lst:javaBinary2}. Since there is
a covariant method parameter the method can not be typed. This implies
that there are no statical checks to prevent the programmer from adding
\emph{SinglyLinkedNode} objects to a \emph{DoublyLinkedNode}.

\begin{lstlisting}[float=ht,caption={Illegal subtyping of binary methods in Java.},label={lst:javaBinary}]
public class SinglyLinkedNode {
	private SinglyLinkedNode next; /*\label{lst:javaBinary0}*/

	public void setNext(SinglyLinkedNode next) {
		this.next = next;
	}

	public SinglyLinkedNode getNext() {
		return this.next;
	}
}

public class DoublyLinkedNode extends SinglyLinkedNode {
	private DoublyLinkedNode prev;
	public setPrev(DoublyLinkedNode prev) {
		this.prev = prev;
		prev.next = this;
	}

	public DoublyLinkedNode getPrev() {
		return this.prev;
	}

	// Overloaded method in Java.
	/*\label{lst:javaBinary2}*/public void setNext(DoublyLinkedNode next) {
			this.next = next;
			next.prev = this;
	}

	public DoublyLinkedNode getNext() {
		/*\label{lst:javaBinary1}*/return (DoublyLinkedNode)this.next;
	}
}
\end{lstlisting}

Listing \ref{lst:break} illustrates what behaviour would be possible
if subtyping would allow covariant parameters. As presented in
Listing \ref{lst:javaBinary} the method \emph{setNext} in the
class \emph{DoublyLinkedNode} expects a \emph{DoublyLinkedNode}
as parameter. Since subtyping allows subsumption, the code in
Listing \ref{lst:break} would type perfectly. But considering
the method \emph{setNext}, the parameter \emph{node} would not
have a field \emph{prev} if the parameter \emph{other} is of type
\emph{SinlgyLInkedNode}.

\begin{lstlisting}[float=ht,label={lst:break},caption={Breaking a doubly linked node.}]
public void breaks(SinglyLinkedNode one, SinglyLinkedNode other) {
	one.setNext(other);
}
\end{lstlisting}

When the subclassing rules are defined they can be formulated type
safely, so that subclasses match their superclasses. With this
assumption, subclasses and superclasses are matching related and
nominal matching is possible for type checking. With this definition
the relation \emph{DoublyLinkedNode} \match \emph{SinglyLinkedNode} holds.

Since matching is a weaker relation than subtyping the assumption of
$a:A$ and $A$ \match $B$ does not imply $a:B$ like subtyping does,
i.e., matching can not help in situations illustrated in Listing
\ref{lst:matchbreak}. The problem in this listing is that matching can
not assure that the parameters \emph{node} and \emph{other} is of the
same type. If this were possible, the problem would be solved.

\begin{lstlisting}[float=ht,label={lst:matchbreak},caption={Matching relation in the break method.}]
public void breaks(node <# SinglyLinkedNode, other <# SinglyLinkedNode) {
	node.setNext(other);
}
\end{lstlisting}

This shows that in this case matching is too weak. Without subsumption,
the type system can not know that \emph{other} is compatible with the
object passed as parameter \emph{node}.

\subsection{Match-Bounded Quantification}
As seen before, Listing \ref{lst:matchbreak} is still not correctly
typeable with the matching relation. It is not enough informative to
know that two variables match the same type to conclude type
subsumption compatibility between two types. But there is another
scenario where matching can show its power: object parametrisation.

The widely used \emph{generics} are formally a universal quantification
subtyping relation which is written as $\forall (X <: A(X))B\{X\}$. The
parametrisation is specified with the universal quantification. The
quantification is further constrained by the type operator $A(X)$. This
expression can be read that the record $B$ can be parametrised with a
type variable $X$ and that the expression is true for all $X$ that are
subtypes of $A(X)$.  Since subtyping is used here to bind \X to a range of
objects it is called f-bounded subtyping. Now it is possible to replace
the subtype relation with matching $\forall(X \mmatch B(X))B\{X\}$
which is denoted as match-bounded quantification. It is the same as
above with the difference that the expression applies for all $X$ that
are matching $B(X)$. Depending on the exact definition of matching two
different interpretations are possible\cite{abadi_subtyping_1996}. The
f-bounded interpretation is further discussed at the end of this
chapter. However, the reason why this is useful is given by the fact
that when parametrisation is used, the type parameter does no longer
obey the matching relation at method parameters and return types.

Here it should be noticed that the quantification can also be done with an
existential quantifier which has a different but also important
interpretation especially if modularisation and information hiding is
considered \cite{cameron_existential_2009,pierce_types_2002}.

\subsection{Type Operators}
\nomenclature{Type Operator}{Function on type level mapping types to
types} A type operator is a function on type level mapping types to
types. Type operators are important for the formal specification of
matching and the understanding of the concepts. In an object-oriented
language the object type can be considered as a record type, e.g.,
an object type with two fields of type \emph{Nat} and \emph{Bool}:
\{a:Nat, b:Bool\}. A record does not have to be fully specified with
concrete types. It is possible to use type variables and bind them
via type operators. A type operator is a function, the lambda-calculus
expression is used, from types to types: \[A_{Op} := \lambda X.A\{X\} \]
where $A := \{a:Nat, b:X\}$ and $A\{X\}$ is $X$ applied to $A$.  However,
objects in object-oriented languages are recursive since they make heavy
use of the \mytype.  Since a function in the lambda calculus can not be
recursive per se a special construct is needed. With the use of a fixed
point\nomenclature{Fix point}{A fix point of a function $f$, regardless of
whether the function is a type operator or a function over $\mathbb{N}$,
is a point where $f(x) = x$. Fix points are important for recursive type
functions} operator it is possible to find a type's fix point without
evaluating an infinite recursive call \cite{gabriel_why_1988}. To
hide the complexity of the recursion theorem the $\mu$-convention
is used. A $\mu$-function is comparable with a $\lambda$-function
with the difference that the parameter will get bound recursively
\cite{pierce_types_2002,simons_theory_2002-3}.

This can now be used to define the fixpoint of the record $A$ and is
written as \[A^* := \mu X.A_{Op}(X).\]

Now, the problem of such a recursive definition is to determine
whether two types are equivalent or not since applying a type
parameter recursively, another record will be created. For example,
after applying $A$ on $A^*$, the type generator creates a new record
which is $\mu X.\{a:Nat, b:\{a:Nat, b:X\}\}$ which clearly differs from
$\mu X.\{a:Nat, b:X\}$. Replacing the type parameter recursively with
the type definition itself is undecidable when the type diverges.
To tackle this problem two approaches can be taken. The first is
to take the equi-recursive approach where both declarations are
interchangeable. The problem is that in the latter approach, infinite
type declarations are possible. The second approach is to define an
explicit type isomorphism\nomenclature{Isomorphism}{An isomorphism is
a bijective map where the map and its inverse are structure-preserving}
\cite{abadi_subtyping_1996}:

\begin{defn}[(Un)folding isomorphism]
	\label{def:foldUnfold}
	\[fold : A\{\mu X.A\{X\}\} \rightarrow \mu X.A\{X\}\]
	\[unfold : \mu X.A\{X\} \rightarrow A\{\mu X.A\{X\}\}\]
\end{defn}

This isomorphism enables to tackle recursive types without infinite type
declarations since it enables the possibility to generate two equal
records from two records where both have different recursive evaluations.

\subsection{Matching as F-Bounded Subtyping}
F-bounded subtyping\footnote{Abbreviates function-bounded subtyping}
is an extension to simple subtyping by adding parametrisation to
types. F-bounded subtyping relation is often called \emph{generics}
and is written as $X$<:$B_{Op}(X)$. It gives the information that $X$
extends $B$. The use of the bound against the type operator of $B$
shows the function in the bound. Considering an example in Java,
Listing \ref{lst:generics} shows how this concept is implemented in a popular
programming language. Since every type that satisfies the f-bound can
be used to replace the type parameter $X$, the formal type of $A$ is
$\forall(X$<:$B(X))A\{X\}$ where the bound is \emph{not} against the
type operator!

\begin{lstlisting}[float=ht,caption={Universal quantified f-bound in Java.},label={lst:generics}]
public class A<X extends B<X>> {
	public void m(X x) {
		x.someMethodOfB();
	}
}
\end{lstlisting}

Here, both cases have to be considered very carefully. In the f-bounded
subtype relation where the bound is a real function, i.e., type operator,
the self reference is not yet bound whereas in generics in Java the self
reference is already bound. As long as the type operator is not evaluated,
the subtyping relation is easier to hold than afterwards. This enables
parametrisation over every extension of an object type and not only over
every subtype of it.

In this case matching is defined as\cite{abadi_subtyping_1996}:

\begin{defn}[Matching as F-Bounded Subtyping]
	\label{def:matchingAsBound}
	\[A \textnormal{\match} B := A <: B_{Op}(A)\]
	\[\forall (X \textnormal{\match} A)B\{X\} := \forall (X <: A_{Op}(X))B\{X\}\]
\end{defn}

The Definition \ref{def:matchingAsBound} shows that matching is the same
as object subtyping where the object is parametrised with a \mytype.
