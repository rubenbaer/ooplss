\part{Theoretical Background}
\chapter{Introduction}
In modern object-oriented programming languages is \emph{subtyping}
the fundamental concept of \emph{subclassing} and inheritance since
subclassing comes hand in hand with subtyping it self. If a subclass
is defined a subtype is defined as well. This part shows that it is
likely as well to have subclasses that are not a subtype of the parent
class since subtyping does not hold between two object types that are
not unnaturally in a programmer's daily work. One of the best known
examples are recursively types with binary methods.

This parts introduces these concepts mentions above and where subtyping
and where subclassing is wanted and where the difference are. Since
subtyping is used for polymorphism in object-oriented languages
subclassing does not. But to make subclassing practical at work an
additional relation needs to be introduced, \emph{matching}. Matching
raises the language to function-bounded typing with recursive self types.

\chapter{Subtyping and Subclassing}
\label{chap:subtypingVsSubclassing}
In this chapter we want you to get familiar with the terms subtyping
and subclassing and their differences. Both concepts exist in some of
the widely known object oriented programming languages like Java, Eiffel
or C++, but are usually mixed up. Either subclassing is supported with
improper subtyping or otherwise.

\section{Subtyping}
Subtyping is a binary relation, denoted with `<:', which defines
a substitution relation between two types and is one style of type
polymorphism. Assuming that a type $A$ is a subtype of $B$, then subtyping
ensures that $A$ has compatible methods and fields to its \emph{supertype}
$B$. However, wherever type $B$ is expected type $A$ can be used.

To ensure this type compatibility between $A$ and $B$ subtyping has the
following properties:
\begin{description}
	\item[Reflexivity] $A$ <: $A$
	\item[Transitivity] $A$ <: $B$ and $B$ <: $C$ imply $A$ <: $C$
\end{description}

Type safe subtyping is defined for objects and fields. Subtyping for objects is
quite easy. If $A$ <: $B$ then $A$ needs at least as many fields as $B$
which are in subtype relation. Further $A$ can contain other fields
which does not have any correspondence in $B$. The relation for fields
differs for methods and state variables.

Variables needs to be type invariant in a subtyping context since
subtyping is not symmetric, i.e. $A$ <: $B$ does not imply $B$ <: $A$
if $A \neq B$.

Subtyping method types, denoted as $A \rightarrow B$ for a function with
domain $A$ and codomain $B$, is a bit more complicated since parameters
and return values have to be considered separately. On one hand the return
type can change covariant with the type hierarchy and on the other hand
the parameters are only allowed to change in contravariant way, i.e.,
change in the hierarchy upwards.

\subsection{Variance}
Variance classifies the type change within the type hierarchy. These
terms are introduced in category theory and were adopted to type theory
\cite{pierce_basic_1991}.

\begin{description}
	\item[Invariance] No type change is allowed.
	\item[Covariance] Type changes are only allowed downwards. If $A$
	<: $B$ and $A'$ <: $B'$ and $B$ has a method with the return
	type of $A'$ then the return type of the same method in $B$
	has to be a subtype of $A'$, e.g., the return type can be $B'$
	or $A'$ because of the reflexivity property of subtyping. As
	example $A' \rightarrow X$ <: $B' \rightarrow X$.
	\item[Contravariance] Contravariance is the opposite of
	covariance. Type changes are only allowed upwards in the type
	hierarchy. The type of function parameters are contravariance,
	e.g., $X \rightarrow A$ <: $X \rightarrow B$ if $B$ <: $A$.
\end{description}

\subsection{Liskov Substitution Principle}
\begin{quotation}
`Let $q(x)$ be a property provable about objects $x$ of type $T$. Then
$q(y)$ should be true for objects $y$ of type $S$ where $S$ is a subtype
of $T$.' -- Barbara Liskov \cite{liskov_behavioral_1999}
\end{quotation}

A formal and strong description of subtyping is given by Barbara
Liskov with her Liskov Substitution Principle, short LSP. This notion
of subtyping implies behavioural subtyping where the correctness
of the given properties should be provable. Since behavioural
subtyping would include a proof system which is undecidability
\cite{poll_behavioural_1998}. However, in languages without this strong
property the type systems can not proof the correctness and this gives
the possibility to formulate programs which may go wrong. With the
strong coupling of subtyping and subclassing in languages like Java,
\cs or C++ the programmer has to decide between code reuse or proper
subtyping. \Cref{sec:sharingTypes} gives a deeper insight of this
problem.

\section{Sharing Types Versus Implementation}
\label{sec:sharingTypes}

Although the term \emph{class} is used in modern class-based object-oriented
languages like Java and C++, the concept of extension is actually the
concept of subtypes and not subclassing. The word type is in the sense of data type, such
as integer, boolean etc., but could also be extended types like objects
in object-oriented languages. They determine the way data is stored and
what operations can be perform on them.

In object-oriented languages like Java and C++, the type hierarchy
is directly linked to the implementation hierarchy, in other words,
inheritance automatically creates a subtype and there is no escape
from this. Of a practical point of view however, inheritance is merely
a mechanism to reuse code in further specialisations of classes. This
leads us to the first conflict: sharing types vs. the implementation. To
understand this have a look at \cref{fig:implementationConflict}
\cite{simons_theory_2003-4}:

\begin{figure}[H]
\center
\begin{emp}[classdiag](20, 20)

Class.Shape("Shapes")()("+move(x, y):Integer", "+drawOn(c: Canvas)");
Class.Ellipse("Ellipse")("f1, f2: Point", "r, s: Integer")();
Class.Rectangle("Rectangle")("o: Point", "w, h: Integer")();
Class.Circle("Circle")("{f1 = f2, r = s}")();
Class.Square("Square")("{w = h}")();

Ellipse.ne = Shape.s + (-10, -20);
Rectangle.nw = Shape.s + (10, -20);
Circle.n = Ellipse.s + (0, -20);
Square.n = Rectangle.s + (0, -20);

drawObjects(Shape, Ellipse, Rectangle, Circle, Square);
link(inheritance)(Ellipse.n -- Shape.s);
link(inheritance)(Rectangle.n -- Shape.s);
link(inheritance)(Circle.n -- Ellipse.s);
link(inheritance)(Square.n -- Rectangle.s);

Class.Point("Point")("x, y: Integer")();

Point.w = Shape.e + (100, 5);
Class.CircleR("Circle")("r: Integer")();
Class.SquareR("Square")("w: Integer")();
Class.EllipseR("Ellipse")("p, q: Integer", "s: Integer")();
Class.RectangleR("Rectangle")("h: Integer")();

CircleR.ne = Point.s + (-10, -20);
EllipseR.n = CircleR.s + (0, -20);
SquareR.nw = Point.s + (10, -20);
RectangleR.n = SquareR.s + (0, -20);

drawObjects(Point, CircleR, SquareR, EllipseR, RectangleR);
link(inheritance)(CircleR.n -- Point.s);
link(inheritance)(EllipseR.n -- CircleR.s);
link(inheritance)(SquareR.n -- Point.s);
link(inheritance)(RectangleR.n -- SquareR.s);

\end{emp}
\caption{Sharing types vs. the implementation}
\label{fig:implementationConflict}
\end{figure}

The left hierarchy expresses the conceptual family: A \emph{Circle}
is a special kind of \emph{Ellipse}, so is \emph{Square} of
\emph{Rectangle}. These objects do not actually `extend' their
parents though, instead, they rather remove properties. For instance, the
\emph{Circle} removes the property of having two radii.	The hierarchy on
the right on the other hand is conceptual nonsense. It shows, however,
how one could want to derive and reuse code, in this case the x and
y coordinates. Because of that, inheritance (and reuse) of code and
subtyping should be treated differently.

\section{Binary Methods and the Problem of Recursive Closure}
\label{sec:recursiveClosure}
Another problem of the subtyping concept in object-oriented languages is
the problem of contravariance: the arguments of a method that is derived
and overwrites the same method in the super type can only have a more
general or same argument type, but not a more specific one:

\begin{figure}[H]
\center
\begin{emp}[classdiag](20, 20)

Class.A("A")()("+method(arg: T')");
Class.B("B")()("+method(arg: T)");

B.e = A.w + (-20, 0);
drawObjects(A, B);
link(inheritance)(B.e -- A.w);

\end{emp}
\caption{Contravariance: the argument type in \emph{B} is more specific than the one in \emph{A}}
\label{fig:contravariance}
\end{figure}

This restriction makes perfectly sense for the type safety in
programming languages. For certain class constellations, it leads to
further problems though.

Assuming there is an object \emph{Animal} that has the method
\emph{mate} that takes an argument of the type \emph{Animal} and
returns a type \emph{Animal}. Intuitively, all other animal types
would be derived from this type, like \emph{Dog} or \emph{Cat} which
would have more specific arguments that accept only values of the type
of itself, so that dogs can only mate with dogs and cats only with
cats\cite{simons_theory_2003-1}. Figure \ref{fig:animalContravariance}
illustrates this situation. Because of the contravariance restriction
in subtyping this cannot be achieved by overriding the method with a
more specific argument type.

\begin{figure}[H]
\center
\begin{emp}[classdiag](20, 20)

Class.Animal("Animal")()("mate(Animal): Animal");
Class.Dog("Dog")()("mate(Dog): Dog");
Class.Cat("Cat")()("mate(Cat): Cat");

Dog.ne = Animal.s + (-20, -20);
Cat.nw = Animal.s + (20, -20);

drawObjects(Animal, Dog, Cat);
link(inheritance)(Dog.n -- Animal.s);
link(inheritance)(Cat.n -- Animal.s);

\end{emp}
\caption{The arguments are more specific, which is not allowed because of the contravariance rule}
\label{fig:animalContravariance}
\end{figure}

A similar problem arises in the method \emph{Object.equals: Object
$\rightarrow$ Boolean}\footnote{For those not familiar with this
notation: Object.equals: Object $\rightarrow$ Boolean means that
the equals method/function takes an Object and returns a Boolean}
in Java. The class \emph{Object} has an implementation of the equals
method that takes an argument of the type \emph{Object}. All objects
automatically are of the type \emph{Object} in Java.	Because of the
constraint of contravariance, it is unsound to override this method
with a more specific argument type which would be reasonable since it does
not make any sense to compare two objects of different types. Instead,
if a programmer wants to implement a new version of this method, the
use of type casts is inevitable.

To get rid of this, a language would need to specialise parameters
covariant. This is possible in Eiffel but this makes the type system
unsafe. Later, a type safe alternative will be presented.

\section{Schizophrenic Self-Reference}
\label{sec:schizoReferences}

In objects there has to be some recursion variable that refers to the
object itself, to be able to call methods or use values on self . In some
languages this is called \emph{self} or \emph{this}, in this explanation,
the term \emph{self} will be used to refer to this variable. \\

There are two main cases when recursion occurs:
\begin{enumerate}
\item When a method accesses methods or properties of the self object
\item When a method has arguments or return values of the same type as the object
\end{enumerate}

The first case is problematic when objects are derived: it is not quite
clear to which object the self variable should refer after the derivation.
In C++ and Java\todo{Java is not a good example since Java knows only
virtual methods}, the self-reference in inherited methods refers to
the base object that was derived. This means that an object may contain
many versions of self-references, which is called \emph{schizophrenic
self-reference}. Other languages like Smalltalk and Eiffel treat the
self-reference differently: they redirect the self-reference to the
derived object. This makes recursion different between these two models.

\begin{figure}
	\centering
	\digraph[scale=0.8]{NaiveRecursion}{
		rankdir=LR;
		c[label=client,shape=plaintext]
		a[label=M,shape=Mrecord];
		b[label=F,shape=Mrecord];
		c->a;
		a->b;
		b->b[tailport=e,headport=s];}\\
	\digraph[scale=0.8]{MutualRecursion}{
		rankdir=LR;
		c[label=client,shape=plaintext]
		a[label=M,shape=Mrecord];
		b[label=F,shape=Mrecord];
		c->a;
		a->b;
		b->a[tailport=s,headport=s];}
	\caption{Naive and mutual recursion}
	\label{fig:schizoRecursion}
\end{figure}

An example of shizophrenic self-reference for the second case could be
a two dimensional vector type \emph{Vec2D} which has two coordinates
and the methods \emph{identity} and \emph{equals}. The identity method
returns the reference to the object itself and the equals method
compares two objects of this type. It therefore takes an argument
of the type \emph{Vec2D}. Assuming there is a subtype \emph{Vec3D}
that inherits all these methods, adds a z-coordinate and overrides the
equals method. Let us further assume that the rule of contravariance
does not apply here. Clearly the equals method of \emph{Vec3D} takes
an argument of the type \emph{vec3D}. Listing \ref{fig:schizoListing}
shows this constellation (in pseudo-code):

\begin{lstlisting}[caption={An example of schizophrenic self-reference},label={fig:schizoListing}]
class Vec2D {
	int x,y;

	Vec2D identity() {
		return self; // A Vec2D self variable
	}

	bool equals(Vec2D vec) {
		return vec.x == self.x && vec.y == self.y;
	}
}

class Vec3D extends Vec2D {
	int z;

	bool equals(Vec2D vec) {
		// Self is of Vec3D type
		if (vec instanceof Vec3D)
			return false;
		// Uses insecure type casting
		retrun super.equals(vec) && self.z == ((Vec3D)vec).z;
	}
}
\end{lstlisting}

In the derived class \emph{Vec3D} the self reference in the equals 
method refers to a \emph{Vec3D} object while the one in the identity method
refers to a \emph{Vec2D} object. This is why an instance of the class
\emph{Vec3D} would be schizophrenic.

This is how the fixation of self-references is done in the subtyping
concept: before the combination, i.e., inheritance, of the objects. In the
concept of subclassing however, it is done after the combination.

\section{Subclassing}

The goal of the concept of subclassing is now to provide a somewhat
different kind of inheritance. With this inheritance, the programmer
should be able to inherit code without automatically creating a subtype.
This solves the problem described in \cref{sec:sharingTypes}. Since
subclassing does not create subtypes, the restriction of contravariance,
described in \cref{sec:recursiveClosure}, is of no relevancy
anymore. And as already mentioned, the fixation of the self references
and types is done after the inheritance, so there will be no schizophrenic
self-references as described in the first case\todo{are they cases clear
structured?} in \cref{sec:schizoReferences}. \\

In subclassing however, the question rises: how should return values and
arguments be typed (see the second case in \cref{sec:schizoReferences})
when defining the methods and fields, since the programmer is not able to
know of what type the subclass will be. This is why there has to be some
sort of placeholder for such cases: the \emph{MyType}. The \emph{MyType}
placeholder is then substituted with the current type after the
inheritance respectively after the inheritance in a subtyping environment.
\todo{Some further explanations are necessary}

\chapter{Making Subclassing Practical}
% estimated pages: 2-3
Subtyping is one of the most fundamental and most well known relation
between recursive types in today's programming languages. In languages
where subclasses not implicitly define subtypes a more general
relation is needed -- \emph{matching} -- which is weaker and does
not support subsumption. Originally \emph{matching} was proposed
as function-bounded subtyping \cite{canning_f-bounded_1989}. Function-bounded
subtyping extends simple subtyping with parametrisation which is popular as
\emph{generics} in well known programming languages like Java or \cs
\cite{barron-estrada_inheritance_2003}. An other interpretation of
\emph{matching} as higher-order subtyping with better properties like
reflexivity and transitivity was shown \cite{abadi_subtyping_1996}.

%% Already coverded above
%\section{Binary Methods}
%A binary method is a method with parameters of the same type as the
%receiver of the method, i.e., the \emph{self} reference in the method. The
%name has the origins in the fact that such methods are binary relations
%mapping to a codomain, e.g., \emph{equal} of type $X \times X \rightarrow
%Boolean$ or \emph{add} of type $Num \times Num \rightarrow Num$. The
%same behaviour can be observed in homogeneous data structures. Typing
%such function in functional languages are straightforward and does not
%rise difficulties. This looks complete different in statically object-oriented
%languages where subtyping is fundamental \cite{bruce_binary_1995}.
%
%\cref{lst:javaBinary} shows how subtyping can not prevent the
%possibility to add objects of type \emph{SinglyLinkedNode} in a
%\emph{DoublyLinkedNode} which is definitely incorrect and has to end
%in a runtime exception. In subtyping existing object state variables
%and methods like \emph{setNext} can not be further specialised. The
%\cref{lst:javaBinary1} shows how the type system has to be escaped
%by casts which is unsafe and.	However, these problems does not rise
%from binary methods them self but in combination with subtyping and
%inheritance since these concepts are associated with subsumption.
%
%\begin{lstlisting}[caption={Illegal subtyping of binary methods in Java},label={lst:javaBinary}]
%public class SinglyLinkedNode {
%	private SinglyLinkedNode next;
%
%	public setNext(SinglyLinkedNode next) {
%		this.next = next;
%	}
%
%	public getNext() SinglyLinkedNode {
%		return this.next;
%	}
%}
%
%public class DoublyLinkedNode extends SinglyLinkedNode {
%	private DoublyLinkedNode prev;
%	public setPrev(DoublyLinkedNode prev) {
%		this.prev = prev;
%		prev.next = this;
%	}
%
%	public DoublyLinkedNode getPrev() {
%		return this.prev;
%	}
%	
%	// Not typeable expression in Java.
%	public setNext(DoublyLinkedNode next) {
%			this.next = next;
%			next.prev = this;
%	}
%
%	// Legal override in the sense of subtyping. Illegal Java;
%	public DoublyLinkedNode getNext() {
%		%\label{lst:javaBinary1})return (DoublyLinkedNode)this.next;
%	}
%}
%\end{lstlisting}
%
%Assuming \emph{DoublyLinkedNode} is a subtype of \emph{SinglyLinkedNode},
%the code in \cref{lst:break} can create a run time exception. As
%long as both parameters are objects of the exact same type the code runs
%without any problem. On the other side if the parameter are from different
%types an runtime error would occur since the parameter of \emph{setNext}
%does not provide a \emph{prev} field.
%
%\begin{lstlisting}[label={lst:break},caption={Breaking a doubly linked node}]
%pubic void breaks(SinglyLinkedNode one, SinglyLinkedNode other) {
%	one.setNext(other);
%}
%\end{lstlisting}
%
%Bruce showed in \cite{bruce_binary_1995} some ways out of such problems
%like avoiding binary methods at all and provide these methods as static
%functions or providing special pair objects providing these binary
%methods. One may argue that it should be possible to change the parameters
%of the objects in a contravariant way, i.e., that \emph{setNext}
%and \emph{getNext} is typed as a method with parameters of type
%\emph{DoublyLinkedNode}. As seen in \cref{chap:subtypingVsSubclassing}
%this would break the subtyping relation between \emph{SinglyLinkedNode}
%and \emph{SinglyLinkedNode}. In this case inheritance provides a subclass
%which is not a subtype.

\section{Matching}
\begin{quotation}
`the same as subtyping in the absence of the \emph{MyType}
construct, but differs in the presence of \emph{MyType}, because
\emph{MyType} implicitly has different meanings in different types.' --
\cite{bruce_foundations_2002}
\end{quotation}

An other solution proposed by Bruce \cite{bruce_binary_1995} is
\emph{matching}. Matching is a generalised subtyping relation, written
as `\match', between two types and takes regard to a \emph{MyType} in
subclasses. An important difference is that the matching relation is only
between object types and not on method types what is in subtyping. As
said by Bruce in \cite{bruce_foundations_2002} matching does not differ
very much from subtyping only that \mytype is not substituted by its
concrete object type. This makes the relation weaker than subtyping
where contravariant method parameters are not allowed.

Listing \ref{lst:javaBinary} contains some problems that can not be tackled
with subtyping. First of all the variable in line \ref{lst:javaBinary0} can
not be specialised since it is invariant. This entails that even when the 
method \emph{getNext} can be specialised in the subclass -- which is not possible
in Java -- an cast operation is necessary to get the correct type in the overridden 
method on line \ref{lst:javaBinary1}.

The second problem is in line \ref{lst:javaBinary2}. Since there is a covariant 
method parameter the method can not be typed. This implies that there are no statically
checks to prevent the programmer from adding \emph{SinglyLinkedNode} objects to a 
\emph{DoublyLinkedNode}.

\begin{lstlisting}[caption={Illegal subtyping of binary methods in Java},label={lst:javaBinary}]
public class SinglyLinkedNode {
	private SinglyLinkedNode next; %\label{lst:javaBinary0})

	public setNext(SinglyLinkedNode next) {
		this.next = next;
	}

	public SinglyLinkedNode getNext() {
		return this.next;
	}
}

public class DoublyLinkedNode extends SinglyLinkedNode {
	private DoublyLinkedNode prev;
	public setPrev(DoublyLinkedNode prev) {
		this.prev = prev;
		prev.next = this;
	}

	public DoublyLinkedNode getPrev() {
		return this.prev;
	}
	
	// Not typeable expression in Java.
	%\label{lst:javaBinary2})public setNext(DoublyLinkedNode next) {
			this.next = next;
			next.prev = this;
	}

	// Legal override in the sense of subtyping. Illegal Java;
	public DoublyLinkedNode getNext() {
		%\label{lst:javaBinary1})return (DoublyLinkedNode)this.next;
	}
}
\end{lstlisting}

Listing \ref{lst:break} illustrates what behaviour would be possible if subtyping 
would allow covariant parameters.

\begin{lstlisting}[label={lst:break},caption={Breaking a doubly linked node}]
pubic void breaks(SinglyLinkedNode one, SinglyLinkedNode other) {
	one.setNext(other);
}
\end{lstlisting}

When the subclassing rules are defined they can be formulated
type safe that subclasses are matching their superclasses. With
this assumption subclasses and superclasses are matching related and
nominal matching is possible for type checking. With this definition
the relation\emph{DoublyLinkedNode} \match \emph{SinglyLinkedNode} holds.

Since matching is a weaker relation then subtyping the assumption of $a:A$
and $A$ \match $B$ does not imply $a:B$ how subtyping does, i.e., matching
can not help in situations illustrated in listing \ref{lst:matchbreak}.

\begin{lstlisting}[label={lst:matchbreak},caption={Matching relation in the break method}]
pubic void break(node <# SinglyLinkedNode) {
	SinglyLinkedNode newNode = new SinglyLinkedNode();
	node.setNext(newNode);
}
\end{lstlisting}

One can see that here is matching too weak. Without subsumption the type system
can not know that \emph{newNode} is compatible with the object passed as parameter.

\subsection{Match-Bounded Quantification}
However, the example in listing \ref{lst:break} is still not correct
typeable as long as only is known that the matching relation between
\emph{DoublyLinkedNode} and \emph{SinglyLinkedNode} holds since assigning
objects to variable is not safe if only is known that the variable's type
matches the object type. This is desirable in homogeneous data structures
but not in heterogeneous. So matching and \mytype together is a good match
for such structures but lacks on flexibility for many other situations.

To tackle such unwanted inflexibility matching can be extended
with type parametrisation, called match-bounded quantification
\cite{abadi_subtyping_1996}. 

\section{Type Operators}
A type operator is a function on type level mapping types to types. Type
operators are important for the formal specification of matching and
understanding of the concepts. In an object-oriented language the object
type can be considered as a record type, e.g., an object type with two
fields of type \emph{Nat} and \emph{Bool}: \{a:Nat, b:Bool\}. If not
every type is fully specified an expression can be typed with a type
variable. For this a type operator is a function, the lambda-calculus
expression is used, from types to types: \[A_{Op} := \lambda X.A(X)
\] where $A := \{a:Nat, b:X\}$. Objects in object-oriented languages are
recursive since they make heavy use of the self-type. However, since a
function in the lambda calculus can not be recursive per se a special
construct is needed \cite{gabriel_why_1988}.  To hide the complexity of
the recursion theorem the $\mu$-convection is used. A $\mu$-function is
to understand similar to a $\lambda$-function with the difference that
the parameter will get bound recursively \cite{simons_theory_2002-3}.

This can now be used to define the fixpoint of $A$ and is written as \[A^*
:= \mu X.A_{Op}(X)\].

\section{Matching as F-Bounded Subtyping}
F-bounded subtyping\footnote{Short form function-bounded subtyping}
is an extension to simple subtyping by adding parametrisation to
types. F-bounded subtyping relation is often called \emph{generics}
and is written as $X$<:$B_{Op}(X)$ and gives the information that $X$
extends $B$. The use of the bound against the type operator of $B$
shows the function in the bound. Considering an example in Java in
listing \ref{lst:generics} shows how this concept is implemented in a popular
programming language. Since every type that satisfies the f-bound can
be used to replace the type parameter $X$, the formal type of $A$ is
$\forall(X $<:$ B(X))A\{X\}$

\begin{lstlisting}[caption={Universal quantified f-bound in Java},label={lst:generics}]
public class A<X extends B<X>> {
	public void m(X x) {
		x.someMethodOfB();
	}
}
\end{lstlisting}

However, this works fine as long as no \mytype is present. If a language contains a \mytype construct matching is inevitable. 


\section{Matching as Higher-Order Subtyping}

\section{Example with Matching}

