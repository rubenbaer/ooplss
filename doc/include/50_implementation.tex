\chapter{Software Design and Implementation}
\label{ctr:swDesignImplementation}

\section{Compiler Design}
This chapter contains a description of the general compiler. Figure
\ref{fig:compilerPhase} shows the different parts of the whole compiler
software. There is a section describing each phase.

\begin{figure}[ht]
	%\centering
	\centerline{\includedot[scale=0.5]{dot/compilerDesign}}
	\caption{Compiler phases}
	\label{fig:compilerPhase}
\end{figure}

The compiler of \ooplss is written in Java using ANTLR as parser
generator. ANTLR supports the generation of code for the lexical and
semantical analysis, which then generates an abstract syntax tree for
the intermediate representation of the input. In ANTLR it is possible to
define both, grammars for string, 1-dimensional data, and for abstract syntax
trees which are 2-dimensional. In ANTLR the latter are known as \emph{tree
grammars} which are similar to normal grammars but matches abstract
syntax tree nodes and not lexer tokens. The lexer and parser part is a
normal grammar whereas the other phases are all tree grammars which are
used for tree rewriting and tree visiting.  The output process
is supported by \st, a project associated to ANTLR.

However, every phase of this compiler design is independent and can
replaced arbitrarily. This makes corrections within this parts easy and
does not automatically break the whole interface between the compiling
phases. Even new phases can be injected between two phases which may
perform some further analysis or code optimisation.

Even though ANTLR supports different output languages for the parser, Java was
chosen and consequently the whole compiler is written in Java. Mixing
the implementation language would unnecessarily increase the effort
by providing a uniform interface from ANTLR code and the implementation
language.

The output translation backend is a translation unit from the
intermediate abstract syntax tree to pure Java code.


\subsection{Lexer and Parser}
The lexer and parser are defined in the same file \emph{Ooplss.g}
since they are strongly connected together.

The lexer creates tokens which are fed to the parser. It simply
analyses every character in the input string and tries to identify tokens
that are used in the parser. Some tokens like comments and whitespace characters
are skipped and not passed to the parser.

The parser parses the tokens from left to right and creates a left most
derivation with a lookahead of two, which is shortly known as LL(2) parsing.
All this is done from top to bottom and produces an
abstract syntax tree (AST)\nomenclature{AST}{Abstract syntax tree which is an
intermediate representation of the input. Opposed to the parse tree, it is
already cleared of unnecessary information and also operator precedence is
already encoded}. To make the debugging process of the parser code simpler,
the lookahead is limited to two even though ANTLR would support arbitrary
lookahead.

After this phase an abstract syntax tree is generated and ready to be visited
and to be operated on. The AST already contains the correct operator precedence.

\subsection{Symbol Table Propagation}
The first part that operates on the abstract syntax tree is the building of
the symbol table. This is divided into two separated compiler runs: the
first identifies every symbol definition like classes, methods and
variables used within a program. Since fields can be declared anywhere
in a class, two separate runs are necessary, one to track all the defined
symbols and one to resolve the types, method calls and variable accesses.

The symbol table has four different scope levels. These scopes are nested
and child scopes can access the symbols in the parent scope.
\begin{description}
	\item[Global scope] All classes and built-in types
	\item[Class scope] All fields of a specific class
	\item[Method scope] The method parameters and variables
	\item[Local scope] Local scopes within a method. These scopes
	constrain the validity of local defined variables and can be
	nested arbitrarily.
\end{description}

To keep the translation to Java simple, the scoping rules are implemented
like in Java. This means that shadowing variables in methods is not possible.
Listing \ref{lst:shadowing} shows this in more detail.

\begin{lstlisting}[float=ht,language=ooplss,caption=Shadowing variables,label=lst:shadowing]

def foo(x:Int) {
	var x:Int; // not possible, shadowing the argument
	var y:Int;
	{ // a local scope
		var y:Int; // not possible, shadowing the variable above
		var z:Int;
	}
	var z:Int; // no problem here
}
\end{lstlisting}

\subsubsection{Built-In Types}
Built-in types are created before walking the tree for the first time.
They are assigned to the global scope, which is created upon creating the symbol
table. The built-in types are documented in the language specification
(\Cref{sec:builtinTypes}).

\subsubsection{Define Symbols}
The first phase of the symbol table building is the \emph{definition run}
or \emph{walk}. Each class, method and scope is visited with the
depth first strategy and every symbol definition within these scopes
is tracked. The tree grammar for this can be found in the file
\emph{OoplssDef.g}. Every time a scope is entered, this scope is set
as the current scope and the one before is saved as the parent scope.
Like this, all symbol definitions are recorded in the according scope.
Upon leaving a class, the current scope pointer is set to
the parent scope, similar like the pop operation on a stack.
Additionally, the current scope is recorded to some AST nodes for later
referencing.

If a symbol is already defined within the current scope and clashes with
an existing one, an exception is thrown.

Upon leaving a class, it is checked whether this class has a constructor.
If not, an empty default constructor symbol is created and assigned to the
class. This simplifies the type checking when creating new instances
of this class.

After this phase has finished, a symbol table was be created for the whole
program where every symbol in every scope is tracked. The table is in reality
a tree of scopes which can resolve every valid symbol.

\subsubsection{Referencing Symbols}
In the second phase, the \emph{referencing run}, mainly every use of a symbol
that is not declarative (method call, variable access, super type specification
, types in declarations etc.) is examined and resolved to the according symbol
definition.  Exceptions are thrown on errors like accessing non-existent
variables, calling variables or using types that do not exist.

All of this is possible due to the fact that now all definitions are known
and available in the tree. Since the current scope was recorded to the AST
nodes, it is clear to the walker where to start resolving the symbols.\\


\noindent There are different types of resolving the symbols:
\begin{description}
\item[Variable referencing]
Just walks up the scope tree to find the symbol
\item[Method calling]
The same as variable referencing but it has to be a method symbol
\item[Member accessing]
Is used when class fields are accessed like x.f; or x.foo();. First it checks
if the field really is accessed on a class. If not, an according exception
is thrown. Otherwise it is checked if a field with such name exists
in the class. Opposed to normal variable referencing, the search does not walk
the scope tree upwards, instead, it tries to resolve the names in the super classes
and super types if not found.
\item[Type resolving]
The same as variable referencing but it also checks whether it's a type or not.
\item[Special type resolving]
Resolves special symbols directly in the global scope
\item[Class resolving (new statement)]
The same as type resolving but it has to be a user defined class
\item[Self resolving]
Resolves to the enclosing class
\end{description}

\noindent The referencing phase also does some special tasks:
\begin{itemize}
\item it ensures that a symbol that was not declared in the class scope is
defined before being accessed for the first time because forward referencing
is only allowed for class fields.
\item upon leaving a class it checks if there are symbols that override symbols
declared in super classes or super types. This cannot be done in the definition
run because only now there is a resolved reference to the super type/class.
In case of a variable override, an exception is thrown since this is not
possible. In case of a method override, the return type and the arguments
are checked, since they have to be the same.
\item when referencing a super type or class it also runs some checks to avoid
cyclic subtyping or subclassing.
\end{itemize}

The result of this phase is a fully propagated symbol table. Every symbol can
now be resolved within this tree with all necessary meta data. The tree grammar
for the reference walker is defined in \emph{OoplssRef.g}

\subsection{Type Checking}

The third walk on the abstract syntax tree does the type checking. This
time the nodes are evaluated in postorder so both sides of an expression
node are evaluated before the node itself is evaluated. Like this the
evaluated types are promoted upwards the tree until a violation of the
typing rules is found.

The basic concept of the type checker is to check if method arguments,
both sides of an expression/assignment etc. have the same type. In case
of basic types this is a bit extended since adding a integer to a float
is perfectly valid. In this case, the type checker yields float as evaluation
type of the addition expression.

In case of method arguments, return statements, new object creation
and assignments it is extended as well since the subtyping concept
allows assigning values that are of a subtype of the expected value.

The typing rules are explained in detail in \Cref{ctr:typeRules}. The
type checker grammar can be found in \emph{OoplssTypes.g}. Most work
is actually delegated to the SymbolTable (\ref{sec:symbolTableDescription}).

\subsubsection{Decidability}
\Cref{ctr:typeRules}

\subsection{Standalone statements}

Since \ooplss is translated into Java, there has to be some checking
done on the statements so the compiler does not create erroneous Java
code. Java has some constraints, for instance meaningless statements
like accessing a variable without assigning it to another variable
or passing it to a function are inhibited by the Java compiler.

We solved this particular problem by assigning every statement whether
it is allowed to stand alone or not. Then the reference walker loops
through all statements upon leaving a block. If one is found that cannot
stand alone, an exception is thrown.

\subsection{Translation}
Translation is the last stage of the compiler backend. It
translates an abstract syntax tree which is type checked to the
output language -- Java. The last tree grammar can be found in the
file \emph{OoplssGen.g}. Again, a tree grammar is used to perform the
translation.

The code translation is template driven, i.e., appropriate templates
were created which cover the used language structures in Java. The
templates are written in \st which is a dynamically typed and functional
programming language. Matches in the abstract syntax tree does directly
call a template which generates appropriate code.

As result the intermediate language representation is translated to the
output Java code which runs on the Java Virtual Machine.

\section{Software Structure}

\subsection{Package Description}

\begin{figure}[H]
\centering
	\begin{emp}[classdiag](20, 20)

	PlainNote.MainN("Main package");
	Package.MainP("ooplss")(MainN);

	PlainNote.AntlrN("Antlr generated Code");
	Package.AntlrP("ooplss.antlr")(AntlrN);
	AntlrP.w = MainP.e + (40, 0);

	PlainNote.SymTabN("The symbol table");
	Package.SymTabP("ooplss.symbolTable")(SymTabN);
	SymTabP.n = AntlrP.s + (0, -40);

	PlainNote.TreeN("Customised AST node");
	Package.TreeP("ooplss.tree")(TreeN);
	TreeP.n = MainP.s + (-40, -40);

	PlainNote.ExceptionsN("Symbol table exceptions");
	Package.ExceptionsP("ooplss.symbolTable.exceptions")(ExceptionsN);
	ExceptionsP.n = SymTabP.s + (0, -40);

	PlainNote.UtilsN("Utils");
	Package.UtilsP("ooplss.utils")(UtilsN);
	UtilsP.w = AntlrP.e + (70, 0);

	drawObjects(MainP, AntlrP, SymTabP, TreeP, ExceptionsP, UtilsP);
	link(subclassof)(MainP.e -- AntlrP.w);
	link(subclassof)(AntlrP.s -- SymTabP.n);
	link(subclassof)(AntlrP.w -- TreeP.n);
	link(doublesubclassof)(TreeP.e -- SymTabP.w);
	link(subclassof)(SymTabP.s -- ExceptionsP.n);
	link(subclassof)(ExceptionsP.w -- TreeP.se);
	link(subclassof)(AntlrP.e -- UtilsP.w);
	link(subclassof)(pathStepX(AntlrP.e, ExceptionsP.e, 40));

	\end{emp}
	\caption{Package diagram}
	\label{fig:packages}
\end{figure}


\begin{description}
\item[ooplss] This is the main package, containing the main routine of the
compiler.
\item[ooplss.antlr] This package contains the code that is generated from the
ANTLR grammars, namely the lexer, parser, definition walker, referencing walker
type checker and the code generator.
\item[ooplss.symbolTable] This package contains the whole symbol table including
all the symbols, types and scopes.
\item[ooplss.symbolTable.exceptions] This package contains all the exceptions
that the symbol table would throw upon finding errors.
\item[ooplss.tree] It was necessary to implement a customised AST node to be
able to add certain information to the nodes. This package contains all the
necessary classes.
\item[ooplss.utils] Contains an error handler to be able to handle the
exceptions thrown by the symbol table.
\end{description}

All of these packages are subpackages of \emph{ch.codedump}.

\subsection{Symbol Table Classes}

Following Figure shows the composition of the classes of the symbol table.
To keep the diagram simple, this is only an abstract representation, i.e. only
the important methods and variables are shown. Also, the variables are actually
private and can only be accessed through getters and setters.

\begin{figure}[H]
	\centering
	\begin{emp}[classdiag](20, 20)
	Class.SymbolTable("SymbolTable")()();

	Class.Scope("Scope")("enclosingScope:Scope")( "define(Symbol):Void",
			"resolve(String):Symbol", "resolveType():Type");
	classStereotypes.Scope("<<interface>>");
	Scope.ne = SymbolTable.sw + (-60, -10);

	AbstractClass.Symbol("Symbol")("def:OoplssAST", "type:Type", "scope:Scope")("");
	Symbol.n = SymbolTable.s + (0, -10);

	drawObjects(SymbolTable, Scope, Symbol);

	Class.BaseScope("BaseScope")()();
	BaseScope.n = Scope.s + (0, -20);

	Class.GlobalScope("GlobalScope")()();
	GlobalScope.n = BaseScope.s + (-40, -20);

	Class.LocalScope("LocalScope")()("define(Symbol):void");
	LocalScope.n = BaseScope.s + (0, -80);

	drawObjects(BaseScope, LocalScope, GlobalScope);
	link(implements)(BaseScope.n -- Scope.s);
	link(inheritance)(LocalScope.n -- BaseScope.s);
	link(inheritance)(pathManhattanY(GlobalScope.n, BaseScope.w));

	AbstractClass.ScopedSymbol("ScopedSymbol")()();
	ScopedSymbol.n = Symbol.s + (0, -30);

	drawObjects(ScopedSymbol);
	link(inheritance)(ScopedSymbol.n -- Symbol.s);
	link(implements)(pathManhattanY(ScopedSymbol.nw, Scope.se));

	Class.MethodSymbol("MethodSymbol")("arguments:List", "origin:MethodSymbol")();
	MethodSymbol.n = ScopedSymbol.s + (-60, -30);
	link(inheritance)(pathManhattanY(MethodSymbol.n, ScopedSymbol.w));

	Class.ClassSymbol("ClassSymbol")
		("supertype:ClassSymbol", "superclass:ClassSymbol", "constructor:MethodSymbol")
		("resolve(String):Symbol", "resolveMember(String):Symbol",
		"resolveSuper(String):Symbol",
		"checkMethodArguments():boolean", "checkMethodReturnTypes():boolean",
		"checkForInheritanceErrors(:boolean)", "checkForOverridings():void",
		"isSubtypeOf(ClassSymbol):boolean", "isSubclassOf(ClassSymbol):boolean");
	ClassSymbol.n = ScopedSymbol.s + (0, -80);

	Class.SuperVariableSymbol("SuperVariableSymbol")()();
	SuperVariableSymbol.e = ClassSymbol.w + (-40, 0);

	Class.Type("Type")()();
	classStereotypes.Type("<<interface>>");
	Type.w = ClassSymbol.e + (40, 0);

	drawObjects(MethodSymbol, ClassSymbol, Type, SuperVariableSymbol);
	link(inheritance)(ClassSymbol.n -- ScopedSymbol.s);
	link(implements)(ClassSymbol.e -- Type.w);
	link(inheritance)(SuperVariableSymbol.e -- ClassSymbol.w);


	Class.VariableSymbol("VariableSymbol")()();
	VariableSymbol.w = Symbol.e + (40, 0);

	Class.BuiltIn("BuiltInTypeSymbol")()();
	BuiltIn.nw = Symbol.se + (40, -30);

	AbstractClass.SpecialSymbol("SpecialSymbol")()();
	SpecialSymbol.nw = BuiltIn.s + (0, -20);

	Class.ConstructorType("ConstructorType")()();
	ConstructorType.n = SpecialSymbol.se + (5, -30);

	drawObjects(VariableSymbol, BuiltIn, SpecialSymbol, ConstructorType);
	link(inheritance)(VariableSymbol.w -- Symbol.e);
	link(inheritance)(BuiltIn.nw -- Symbol.s);
	link(implements)(pathStepY(BuiltIn.sw, Type.nw, -70));
	link(inheritance)(pathStepX(SpecialSymbol.e, Symbol.se, 10));
	link(implements)(pathStepY(SpecialSymbol.sw, Type.n, -20));
	link(inheritance)(pathStepY(ConstructorType.n, SpecialSymbol.s, 5));

	\end{emp}
	\caption{Classes of the symbol table}
	\label{fig:classes}
\end{figure}

\subsection{Symbol Table}
\label{sec:symbolTableDescription}

The class \emph{SymbolTable} is the heart of the whole symbol table structure.
On creating the symbol table, it creates the global scope and the built-in
types and assigns the types to the global scope. The global scope is the root
node for the scope tree. Most operations triggered by the definition and
referencing tree grammars go through some method in \emph{SymbolTable}.
The operations can be grouped into 3 main kinds of operations:

\paragraph{Simple type checking}
These methods perform simple type checking on two AST nodes in case of
arithmetic (+ - * /), relational (< > <= =>) and equality (== !=) operators.
There are three tables respectively containing all the possible combinations
between the types, whereat these types are either one of the built-in types,
\emph{object} in case of a class or \emph{mytype}. These tables contain the
result type of the operation with the two given types. For instance, adding
an integer to a float yields the type float. If the result type
is \emph{void} it means that these types cannot be used with the given
operation and an exception is thrown.

Additionally there is a method to check whether a condition is of
the type \emph{bool}. This is used for if and while statements.

\paragraph{Object type checking}
In case of variable assignment, method arguments and return statements,
a more sophisticated type checking is needed because of the possibilities
of subtyping and subclassing. These methods take care of the checking
of those cases.

\paragraph{Name resolving}
There are several methods to resolve different kinds of symbols, namely
types, classes etc. These methods cover all these different kinds and their
specialities and prevent from using the symbols in a wrong way.

%\section{Translation}
%\label{ctr:translation}
%This chapter deals with the exact translation from \ooplss code segments
%to Java.
%
%Done to java as seen in part III
%
%\todo{Export code and intermediate code in X\$\$...}

\section{Unit Testing}
\subsection{gUnit -- A Grammar Testing Framework}

To be able to run unit tests on the grammar, the gUnit\footnote{
\href{http://www.antlr.org/wiki/display/ANTLR3/gUnit+-+Grammar+Unit+Testing}
{http://www.antlr.org/wiki/display/ANTLR3/gUnit+-+Grammar+Unit+Testing}}
feature of ANTLR is used. gUnit tests can be run directly on the grammar files.

There are two types of testing: simple grammar checks and AST checks. The
first simply checks whether the input yields an error or not. This is useful
to test if valid input code is accepted and erroneous code rejected. The
second type is to check whether the AST tree is rewritten correctly. It is
possible to write the expected AST for every rule in the grammar.\\

The gUnit tests can be found in \emph{grammar/test/}.


\subsection{JUnit}

The symbol table is checked via JUnit tests. There are various tests that
test the symbol table in a certain depth, i.e. the definition walker, the
reference walker and the type checker.

For now these unit tests are rather rudimentary, they can only check whether
code is run without errors or with expected exceptions. To test everything
properly, it is also required to watch the debugging output.
If there were more time, we would implement unit testing methods that can
actually examine the created contents and state of the symbol table or the
generated Java output.


\subsection{Examples}
see \Cref{ctr:exampleCode}
% estimated pages: 5
