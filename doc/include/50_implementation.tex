\chapter{Implementation}
\section{Design}
This chapter contains a description of the general compiler. Figure
\ref{fig:compilerPhase} shows the different parts of the whole compiler
software. For every phase exists a corresponding section.

\begin{figure}[ht]
	%\centering
	\centerline{\includedot[scale=0.5]{dot/compilerDesign}}
	\caption{Compiler phases}
	\label{fig:compilerPhase}
\end{figure}

The compiler of this prototypical language is written in Java using
ANTLR as parser generator. ANTLR support the generation of the lexical
and semantical analysis which generates an abstract syntax tree which
is denoted as intermediate representation. In ANTLR it is possible to
define both, grammars for string , 1-dimensional, data as well as for
abstract syntax trees which are 2-dimensional. In ANTLR these are known
as \emph{tree grammars} which are similar to normal grammars but matches
abstract syntax tree nodes and not lexer tokens. The lexer and parser
part is a normal grammar whereas the other phases are all tree grammars
which are used for tree rewriting. The output process is supported by
\st, a project associated to ANTLR.

However, every phase of this compiler design is independently and can 
replaced arbitrary. This makes corrections within this parts easy and
does not automatically breaks the whole interface between compiling
phases. Even new phases can be injected between two phases which may 
perform some further analysis or code optimisation.

Even ANTLR supports different output languages for the parser, Java was
chosen and consequently the whole compiler is written in Java. Mixing
the implementation language would unnecessarily increase the effort
by providing a uniform interface from ANTLR code ant implementation
language.

The output translation back end is a translation unit from the
intermediate abstract syntax tree to pure Java code.

\subsection{Lexer and Parser}
The lexer and parser are both in the same file \emph{Ooplss.g} defined
since they are strongly connected together. 

The lexer creates token which are feed to the parser. It does simply
analyse every character in the input string and tries to identify token
which are used in the parser. The comment and whitespace tokens are all
skipped and will not get forwarded to the parser phase.

The parser parses the token from left to right and creates a left most
derivation with a lookahead of two, this is shortly known as LL(2) parser.
All this is done from top to bottom and produced from the token read an 
abstract syntax tree. For the sake of simplicity during the debugging
process of the parser code the lookahead is limited to two while ANTLR
would support arbitrary lookahead.

After this phase an abstract syntax tree is generated and ready to be visited
and translated to other trees and output code.

\subsection{Symbol Table Deployment}
The first part that works on the abstract syntax tree is the deployment of
the symbol table. This is divided into two separated compiler runs. The
first identifies every symbol definition like classes, methods and
variables used within a program. Since symbols in object-oriented
languages are context sensitive and can be recursive, this definition
of two phases is inevitable unless forward declarations are provided
which are well known in C++.

The symbol table has four different scope levels. These scopes are nested
and child scopes can access the symbols in the parent scope.
\begin{description}
	\item[Global scope] All classes and basic types
	\item[Class scope] All fields of a specific class
	\item[Method scope] The method parameters and variables
	\item[Local scope] Local scopes within a method. This scopes
	define the validity of local defined variables.
\end{description}

\subsubsection{Define Symbols}
This is the first phase of the symbol table deployment. Every single
class, method and scope is visited with the depth first strategy and
tracks every symbols definition and skips the use of the symbols. The
tree grammar for this can be found in the file \emph{OoplssDef.g}.

If a symbol is already defined within the current scope and clashes with
an existing one, an exception is thrown.

After this phase a symbol table is created for the whole program where
every symbol in every scope is tracked. The table is in reality a tree
which can resolve every valid symbol within a scope.

\subsubsection{Referencing Symbols}
This is the second phase of the symbol table promotion. Again, every scope
is visited. However, in this run, the type definitions are resolved and
promoted to the definition under consideration. Further more the symbols
used within the code are resolved and does not reference to the correct
instance within the symbol table. The tree grammar code can be found in
the file \emph{OoplssRef.g}

All of this is possible since now all definitions are known and available
in the tree.

The result of this phase is a fully propagated symbol table. Every symbol can
now be resolved within this tree with all necessary meta data.

\subsection{Type Checking}
\todo{TODO!}
\subsubsection{Decidability}
\Cref{ctr:typeRules}

\subsection{Translation}
Translation is the last stage of the back end from the compiler. It
translates an abstract syntax tree which is type checked to the
output language -- Java. The last tree grammar can be found in the
file \emph{OoplssGen.g}. Again, a tree grammar is used to perform the
translation.

The code translation is template driven, i.e., appropriate templates
were created which cover the used language structures in Java. The
templates are written in \st which is a dynamically typed and functional
programming language. Matches in the abstract syntax tree does directly
call a template which generates appropriate code. The detailed syntax
equivalence of \ooplss to Java is described in \Cref{ctr:translation}.

As result the intermediate language representation is translated to the
output Java code which runs on the Java Virtual Machine.

\section{Translation}
\label{ctr:translation}
This chapter deals with the exact translation from \ooplss code segments
to Java.

Done to java as seen in part III
% estimated pages: 10

\todo{Export code and intermediate code in X\$\$...}

\section{Tests}
\subsection{Unit Testing}
\subsubsection{GUnit -- A Grammar Testing Framework}
\subsubsection{JUnit}

\subsection{Examples}
see \Cref{ctr:exampleCode}
% estimated pages: 5
