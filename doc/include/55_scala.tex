\chapter{Comparison with Scala}
\label{sec:comparisonScala}
This chapter presents a comparison of some features of Scala and \ooplss.
One example is about typing the \self-variable in subclassing for
\ooplss and explicitly typing in Scala. A second one treats the
issue of software composition. Scala provides a flexible construct called
traits which has some similarities with subclasses in \ooplss in regard to
code reuse.

\section{Scala -- A Scalable Language}
Scala is a modern programming language initiated by Martin Odersky,
director of \emph{Laboratoire de Méthodes de programmation} at the
\emph{école polytechnique fédérale de Lausanne}. It is a general
purpose programming language with special remarks on scalability
and flexibility without dropping the advantages of a static type
system. Scala provides a multi-paradigm language with support for
object-oriented and functional programming. As target platform both JVM
and CLR\footnote{Commmon Language Runtime is the run-time environment
for Microsoft's .NET technology} can be chosen by the programmer. These
platforms are seamless integrated into Scala which enables the use of
existing software and libraries available on these platforms.

During the development of Scala, a special focus was the type system.
It contains many different features that are not or
only rarely implemented in popular languages with static type systems
like abstract type definitions in abstract classes, explicitly typed self
references and traits, software units for proper code reuse. It is also
remarkable that Scala provides a nominal as well as a structural
type system. These are however not all the special features within
Scala; further information is available in
\cite{odersky_scala_2010,odersky_programming_2008} or on its website.
\footnote{\href{http://www.scala-lang.org/}{www.scala-lang.org}}

Since Scala combines a wide range of features in its type system it is
an optimal language for a comparison of some type features of \ooplss
with those of Scala, providing a hands on how a problem can be solved
with different approaches.

\section{Self-Reference in Covariant Changes}
%Abtract Types and Explicit Self Typig
%Explicit self typing
A common concept which is available in many popular object-oriented
programming languages is the definition of abstract classes which provide
a partial implementation of a class. Such classes can not be instantiated
by the user, but they can be specialised to concrete classes with a full
implementation. In many implementations of this feature, e.g., in \cpp,
\cs or D, only abstract methods and properties are supported. Scala
goes one step further, adding abstract type definitions. An abstract
type can be simply used as normal type declaration within this class. At
the last moment where the abstract class is derived to a concrete class,
these type declarations have to be specified with an existing type, i.e.,
an abstract type declaration is a type parameter that is not exposed
as generics are. However, in the following Scala example, abstract
types are used since they are more compact in this case. Further, inner
classes are used; as in Scala it is possible to nest classes within others.

\subsection{A List in Scala}
Listing \ref{lst:abstractType} presents an implementation of a List
in Scala. It is not the implementation of a possible list, but it
demonstrates how Scala uses explicitly typed self references and abstract
types. It shows how Scala deals with covariant method parameters and
how self-references can be explicitly typed.

The example is an adaption of the doubly linked node which was
introduced in \Cref{subsec:matching}. It contains three global classes,
each containing one inner class. The outer classes implements lists and
the inner classes are the nodes for each list implementation. The nodes
contain the values.  Similar to the implementation of the doubly linked
list, a singly linked list would be possible as well.

The first class is used as an abstract interface with a minimum of fields.
On lines \ref{line:abstractType1} and \ref{line:abstractType2} there are
abstract type declarations. The one on line \ref{line:abstractType1}
specifies a type constraint.  In the second class a partial
implementation is done, but it so far remains abstract since a concrete
node implementation is still missing. Line \ref{line:factory} defines
an abstract factory method for the node
creation. The factory method needs to create a new node and insert the
node the given index. Still in the class \emph{AbstractList} the abstract
types are used. Here the factory method expects two parameters of type
\emph{Node} and \emph{Value} respectively. In its inner class \emph{AbstractNodeImpl},
the factory method is used on line \ref{line:useFactory} where the
self variable is passed to this method. And here the problem of the self
reference appears, since it can not be compatible with the abstract type defined
later in the code. However, let us first take a look at the last and concrete class
of the list; \emph{DoublyLinkedList}. On line \ref{line:typeInstance1}
and \ref{line:typeInstance2} the abstract types are instantiated with
a concrete type. \emph{Node} has now the type \emph{DoublyLinkedNode} and
\emph{Value} is of the type \emph{ValueType}, which is a class type parameter. Further, all
remaining methods and classes are implemented using a doubly linked node.
In a singly linked list, the method \emph{getPrev} would be more complex
in time.

Now let's consider the problem on line \ref{line:useFactory}. As
mentioned, the abstract type \emph{Node} is now of the concrete
type \emph{DoublyLinkedNode}. Consequently the factory
method \emph{createNode} expects a parameter value of type
\emph{DoublyLinkedNode}. What now happens is that this factory has a
covariant type change in its parameters which does not conform with the
subtyping rules. To overcome this problem, Scala allows to type the
self reference explicitly which is done on line \ref{line:explicitSelf}.
Here \emph{self} is now typed with the abstract type \emph{Node}. Without this
line the Scala compiler returns with a typing error since \emph{self} is not
compatible with \emph{DoublyLinkedNode} even though the run-time
type of self is correct in this example. Using this, unsafe structures
like casts can be avoided.

\lstinputlisting
[language=scala,caption=Abstract type declaration in Scala.,label=lst:abstractType]
{../misc/sources/scalaList/List.scala}

\subsection{A List in \ooplss}
Based on the example code in Scala in the previous section, a corresponding
implementation of the doubly linked list in \ooplss will be given in Listing
\ref{lst:doublyLinkedList}. It is assumed that the whole specification
of \ooplss is implemented in the compiler since the examples makes often use
of class parametrisation.

Analogue to the solution written in Scala, the same classes exist in
\ooplss. Since \ooplss does not have inner and abstract classes, providing
a solution for the example, it is costlier than in Scala. This is because
of the following reasons: For every method a block is obligatory, even
though the implementations do not make sense since they only return
\emph{null}. Every node class has its own type parameters since they can
not be derived from the outer classes. This makes the usage a bit more
difficult since the node has to call the factory method \emph{createNode}
which is defined within the list class. However, these drawbacks could
be solved by allowing inner and abstract classes, which was not taken
into consideration within this thesis project, would however be possible.

The major differences are that the classes are derived with
\emph{subclassing} and not with \emph{subtyping}, so the type
\emph{Node} is bound by matching and not by subtyping, and the
\mytype is used within the node classes. Let us consider the line
\ref{line:ooplssUseFactory}. This was the construct which caused problems
in Scala. In \ooplss however, this is perfectly typeable because on
line \ref{line:listMyType} a list is specified which uses \mytype as
type parameter. This ensures that the list uses the correct node type,
e.g., a singly linked node or doubly linked node in singly
linked lists or doubly linked lists, respectively.

\lstinputlisting
[language=ooplss,caption=A doubly linked list in \ooplss.,label=lst:doublyLinkedList]
{../misc/sources/ooplssList/List.ooplss}

\section{Software Composition}
One of the largest problems in designing software is how to compose
software from smaller parts to larger ones, where every statement
meets the requirement of being type safe, comprehensible, reusable and
changeable. For this, many different approaches and methodologies were
developed.

One method of object combination is multiple inheritance
\cite{taivalsaari_notion_1996}. Although there are no objections
to multiple inheritance, it is difficult to use it well
\cite{wegner_panel_1987}. The problem is that inheritance of code
and code reuse is mixed within the same class combination. Furthermore,
it is often said that multiple inheritance is too complex for proper use.
This statement can be supported when the trend of current programming languages
is observed. Java and \cs completely give up multiple inheritance for
the sake of simplicity and introduce interfaces. However, this makes
code reuse more difficult since the programmer can not simply restructure
a class to smaller pieces without changing the type semantics.

To overcome this problem some approaches were developed which
all have the target to build software units that can be easily be
combined into new ones. Already in the end of the 80s the concept of
\emph{mixins}\nomenclature{Mixin}{A class that can be reused by a
subclass. A separate instantiation is not possible} was introduced
in a Lisp extension \cite{moon_object-oriented_1986}. The benefit
of this semantics is that mixins do not define subtypes, which is
done in multiple inheritance. A further development of mixins are
traits\nomenclature{Trait}{An extended model of mixins for composable
software units}. Traits differ in the sense that they provide a more
sophisticated combination semantics which contains renaming and shadowing
\cite{schaerli_traits:_2003}. The reason is that traits use flattening
when combining multiple traits in a full type hierarchy where mixins
use linearisation. The next section presents how Scala ensures code
reusability with traits, whereas \ooplss does so with subclasses.

\subsection{Class Composition with Traits in Scala}
In this section a new example is introduced. It shows how reusable code
segments for ordering can be defined.

Listing \ref{lst:scalaTraits} contains a definition for two traits in
Scala. The first one defines a strictly ordered relation. The type
parameter is explicitly named \mytype. This type parameter should
not be mixed up with the one of \ooplss, but, in this example the
type parameter is used similarly to the \mytype in \ooplss. This
trait leaves the implementation of \emph{compare} abstract for later
implementation. Within this trait, two relations are defined. One is the
\emph{greater} and the other the \emph{less} than relation. This trait
is then further specialised to the ordered trait introducing a non strict
ordering. This is done uniformly. Together they are the same as the Scala
provided \emph{Ordered} trait which is available in the preload package.

After line \ref{line:literatureScala} three classes are provided. The
first one is an abstract class defining literature containing the author
and the year of the publication. Object instances of this class can not
get ordered since it does not use the traits already defined. Based on
this class two others are derived: \emph{Book} and \emph{Journal}. Now
both classes contain the ability to get in order. But what would happen,
if \emph{Literature} would already be defined as orderable? Would it be
possible for \emph{Book} and \emph{Journal} to specialise the parameters
of the ordered traits? This has to be denied. If \emph{Literature} would
be defined as orderable, the other trait extensions would clash with
the one from their superclass since the type parameters are different.

\newpage
\lstinputlisting
[language=scala,caption=Code reuse with traits in Scala.,label=lst:scalaTraits]
{../misc/sources/scalaTraits/Traits.scala}

\subsection{Class Composition with Subclassing in \ooplss}
In this section, a solution to the previous problem with \ooplss is proposed in Listing
\ref{lst:ooplssSubclasses}. The difference to the solution with
Scala is that no type parametrisation is necessary since this happens
implicitly by \mytype. Since \ooplss does not know traits and abstract
class, every class is fully defined.

However, subclassing is used here as derivation and the class
\emph{Literature} already contains the definitions for ordering.

\lstinputlisting
[language=ooplss,caption=Code reuse with subclasses in \ooplss.,label=lst:ooplssSubclasses]
{../misc/sources/ooplssSubclasses/Subclasses.ooplss}

\section{Conclusion}
In this chapter two comparison examples were presented. The first was a
list implementation and the second an ordered library. Both presented
different difficulties within software composition and typing. Even
though \ooplss, as a prototypical language, can not match up to Scala,
these examples show that some problems in other languages can be tackled
by Scala as well as by \ooplss. With the possibility of subclassing in
\ooplss a mechanism for software combination is introduced which can be
used very powerfully in suitable situations.
