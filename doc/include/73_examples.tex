This chapter presents some code examples written in \ooplss.  To make an
example executable, at least the class \emph{Application} is necessary
containing the method \emph{run(): Void}. After the compilation, the program
can be executed using Java. The application name is always \emph{App}.

\section{Executable Hello World}
\lstinputlisting
[float=h,language=ooplss,caption=A minimal hello world example with basic input/output.,label=lst:exHelloWorld]
{../misc/sources/ooplssHelloWorld/HelloWorld.ooplss}

\newpage
\section{Vectors}
\lstinputlisting
[float=h,language=ooplss,caption=A vector for 2D and 3D coordinates.,label=lst:exVectors]
{../misc/sources/ooplssVectors/Vectors.ooplss}

\newpage
\section{Animal}
\lstinputlisting
[float=h,language=ooplss,caption=Refined Animal with Cats and Dogs.,label=lst:exAnimal]
{../misc/sources/ooplssAnimals/Animal.ooplss}

\newpage
\section{MVC Framework}
This example will not build since the compiler does not support
type parametrisation which is used.

\lstinputlisting
[float=h,language=ooplss,caption=A small MVC framework using subclassing.,label=lst:exMVC]
{../misc/sources/ooplssMVC/MVC.ooplss}
