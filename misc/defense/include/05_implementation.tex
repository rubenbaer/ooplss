\section{Implementation}
\begin{frame}[t]{Implementation}
	\note{Heine\\}
	\note{Dauer 7min\\}

	\begin{bigitemize}[<+->]{3mm}
		\item Lexer / Parser
		\note[item]<1> {Zuerst Grammatik für Lexer / Parser}
		\item AST Generierung
		\note[item]<2> {Daraus einen Abstract Syntax Tree generieren}
		\note[item]<2> {AST = Bereinigt von unnötigen Input-Symbolen}
		\note[item]<2> {Operationen mit richtigem Vorrang}
		\item Aufbau der Symbol Table
		\note[item]<3> {Eine Struktur mit allen Symbolen, die im Input-Code vorkommen}
		\item Type Checking
		\note[item]<4> {Auflösen der Symbole}
		\note[item]<4> {Überprüfen der Typenkompatibilität}
		\item Übersetzung
		\note[item]<5> {Wenn sicher ist, dass der code stimmt, Übersetzung}
	\end{bigitemize}

\end{frame}

\subsection{Technologien}
\begin{frame}[t]{Technologien}
	\begin{bigitemize}[<+->]{3mm}
		\item Parser Generator
		\note[item]<1> {Begrenzte Zeit, zuviel Aufwand wenn selber machen}
		\note[item]<1> {Tool zum Erstellen von einem Parser}
		\note[item]<1> {Entscheidung zwischen ANTLR und JavaCC}
		\begin{itemize}
			\item ANTLR (\textbf{AN}other \textbf{T}ool for \textbf{L}anguage \textbf{R}ecognition)
			\note[item]<2> {ANTLR=Another Tool for Language Recognition}
			\note[item]<2> {Wahl, weil es enthält alles}
			\note[item]<2> {JavaCC benötigt weitere Tools}
		\end{itemize}

		\item Java 
		\note[item]<3> {Programmiersprache Java gewählt}
		\begin{itemize}
			\item Implementierungssprache 
			\note[item]<4> {Sprache für die Implementation}
			\note[item]<4> {Hauptsprache von ANTLR nebst anderen Zielsprachen}
			\item Zielsprache
			\note[item]<5> {Zielsprache nach Übersetzung auch Java}
			\note[item]<5> {Gut bekannt bei beiden Studenten}
		\end{itemize}
	\end{bigitemize}
\end{frame}

\subsection{Symbol Table}
\begin{frame}[t]{Symbol Table}
	\begin{bigitemize}[<+->]{3mm}
		\item Bereitstellen der möglichen Symbole und Typen
		\note[item]<1> {Bereitstellen der möglichen Symbole und Typen}
		\note[item]<1> {Klassensymbole}
		\note[item]<1> {Methodensymbole}
		\note[item]<1> {Variablen}
		\note[item]<1> {Built-In types}

		\item Aufbau der Scopes 
		\note[item]<2> {Scope-Aufbau}
		\note[item]<2> {Stack-Mässiger Aufbau}
		\note[item]<2> {Beim betreten eines Blocks push}
		\note[item]<2> {Beim verlassen pop}

		\item Auflösen von Symbolen
		\note[item]<3> {Auflösen der Symbole}
		\note[item]<3> {Überprüft dabei gleich das vorhandensein}

		\item Type Checking
		\note[item]<4> {Alle Methoden für's Type-Checking}
	\end{bigitemize}
\end{frame}

\subsection{ANTLR Grammatiken}
\begin{frame}[t]{ANTLR Grammatiken}
	\begin{bigitemize}[<+->]{3mm}
		\item Lexer / Parser
		\note[item]<1> {Grammatik für den Lexer und Parser}
		\note[item]<1> {Rewrite-Rules für den AST}

		\item Definition Run
		\note[item]<2> {Definition Run}
		\note[item]<2> {Grammatik um den AST zu matchen}
		\note[item]<2> {Baut den Stack auf}
		\note[item]<2> {Legt alle Symboldefinitionen in der Symboltable ab}
		\note[item]<2> {Speichert die Scopes im AST}

		\item Referencing Run
		\note[item]<3> {Referencing Run}
		\note[item]<3> {Versucht alle Symbole aufzulösen, da jetzt alle bekannt sind}
		\note[item]<3> {Im Falle von unbekannten Symbolen Exception}

		\item Type Checker
		\note[item]<4> {Type checking}
		\note[item]<4> {Im Grund Vergleich links/rechts}
		\note[item]<4> {Problematisch war vor allem MyType-Bindung}
		\note[item]<4> {Problematisch waren Methodenargumente, visited bevor konkreter Type bekannt war}
		\note[item]<4> {Besteht aus eigentlich 2 files und damit 2 runs}

		\item Übersetzung
		\note[item]<5> {Am Schluss die Übersetzung}
		\note[item]<5> {Mittels String Templates}
		\note[item]<5> {Problematiken beim }
		\note[item]<5> {\textbf{TODO}}
	\end{bigitemize}
\end{frame}

%	\note{Betonen, was nicht erwähnt wurde in Doku\\}
%	\note[item]{Was wurde implementiert}
%	\note[item]{Wie wurde implementiert}
%	\note[item]{Schwierigkeiten und Probleme}
%	\note[item]{Verwendete Technologien (Warum ANTLR3)}
%	\note[item]{Testing?}
%\end{frame}
