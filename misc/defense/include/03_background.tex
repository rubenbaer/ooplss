\section{Hintergründe und These}
\begin{frame}{Hintergründe und These}
	\note[item]{Begriffe}
	\note[item]{Nominal / Structural Type System}
	\note[item]{Subtyping}
	\note[item]{Subclassing}
	\note[item]{Matching}
	\note[item]{These}
	\begin{bigitemize}{3mm}
		\item Begriffe
		\item Subtyping
		\item Subclassing
		\item Matching
		\item These
	\end{bigitemize}
\end{frame}

\subsection{Beegriffe}
\begin{frame}{Begriffe}
	\begin{description}
		\item[Nominal Type System] Kompatibilität von Typen wird explizit deklariert
		\item[Structural Type System] Kompatibilität wird implizit durchgeführt
		\item[Derivation] Beinhaltet beides, \emph{Extension} und \emph{Inheritance}
		\item[Extension] Generiert Subtypen
		\item[Inheritance] Generiert Subklassen
		\item[Subtyping] Typenrelation für Typen-Polymorphimus
		\item[Matching] Schwächere Typenrelation als Subtyping
		\item[Subclassing] Klassenerweiterung für Matching
	\end{description}
	\note[item]{Subtyping}
	\note[item]{Subclassing}
	\note[item]{Matching}
\end{frame}

\subsection{Subtyping}
\begin{frame}{Subtyping}
	\begin{bigitemize}[<+->]{3mm}
		\item Wird mit `$<:$' gekennzeichnet
		\note[item]<1>{Bezeichnung von Subtype relation}
		\note[item]<1>{In nominal type systems wird es explizit definiert}

		\item Ist transitiv und reflexiv
		\note[item]<2>{Transitiv und reflexiv}

		\item Definiert auf Objekten, Funktionen und Variablen
		\note[item]<3>{Objektrelation, subtype muss mehr Felder haben. Alle Superfelder müssen korrespondierende Subtypen Felder haben}
		\note[item]<3>{Funktionenrelatio, Parameter sind contravariant und Returntypen sind Covariant}
		\note[item]<3>{Variablen, sind invariant}

		\item Unterstützt Typenpolymorphie
		\note[item]<4>{Typen polymorphie}

		\item Problem mit rekursiven Typen
		\note[item]<5>{Problem mit rekursiven typen}

		\item Typen unsichere binäre Methoden
		\note[item]<6>{Typenunsicherheit bei binären Methoden}
	\end{bigitemize}
\end{frame}

\subsection{Subclassing}
\begin{frame}{Subclassing}
	\begin{bigitemize}[<+->]{3mm}
		\item Offener als Subtyping
		\begin{itemize}
			\item<1-> Kovariante Methodenparameter
			\note[item]<1>{Kovariante Methodenparameter}

			\item<1-> Kontravariante Rückgabetypen
			\note[item]<1>{Kontravariante Rückgabetypen}

			\item<1-> Kovariante Variablen
			\note[item]<1>{Kovariante Variablen}
		\end{itemize}

		\item Subtyping ist nicht mehr gegeben
		\note[item]<2>{Als konsequenz, kein Subtyping mehr}

		\item Mögliche Umsetzung mit Generics
		\note[item]<3>{Mögliche Umsetzung mit Generics}
		\note[item]<3>{Problem bei rekursiven Typen, e.g., binären Methoden}

		\item Problem mit rekursiven Typen
		\note[item]<4>{Problem mit rekursiven typen}

		\item Typen unsichere binäre Methoden
		\note[item]<5>{Immer noch Typenunsicherheit bei binären Methoden}
	\end{bigitemize}
\end{frame}

\subsection{MyType}
\begin{frame}{\mytype}
	\begin{bigitemize}{3mm}
		\item Impliziter Typenparameter
		\note[item]{Impliziter Typenparameter}

		\item Typensicherheit für rekursive Typen!
		\note[item]{Typensicherheit für rekursive Typen}

		\item Typensicherheit für binäre Methoden!
		\note[item]{Typensicherheit für binäre Methoden}
	\end{bigitemize}
\end{frame}

\subsection{Matching}
\begin{frame}{Matching}
	\begin{bigitemize}[<+->]{3mm}
		\item Wird mit `\match' gekennzeichnet
		\note[item]<1>{Bezeichnung von Match relation}
		\note[item]<1>{In nominal type systems wird es explizit definiert}

		\item Reflexiv und transitiv
		\note[item]<2>{Reflexiv und transitiv}

		\item Schwächer als Subtyping
		\note[item]<3>{Schächer als Subtyping}

		\item Nur auf Objekttypen definiert
		\note[item]<4>{Nur auf Objekttypen definiert}

		\item Unterstützt keine Typenpolymorphie
		\note[item]<5>{Keine Polymorphie}

		\item Unterstützt den \mytype Parameter
		\note[item]<6>{Unterstützt den \mytype}
	\end{bigitemize}
\end{frame}

\subsection{These}
\begin{frame}[c]{These}
	\begin{quote}
	Die Modell von Subclassing kann simultan zu einem Subtyping-Modell
	in einem einzigen Modell vereint werden, wobei rekursive Typen mit
	einem impliziten Typenparameter, \mytype, dargestellt werden kann.
	\end{quote}
\end{frame}
