\section{Einleitung}
\subsection{Aufgabenstellung}
\begin{frame}[t]{Aufgabenstellung}
	\begin{bigitemize}[<+->]{3mm}
		\item Programmiersprache
		\note[item]<1> {Eine objektorientierte Programmiersprache entwickeln}

		\item Subtyping
		\note[item]<2> {Programmiersprache muss Konzept subtyping beinhalten}
		\note[item]<2> {Bekannt bei aktuellen Sprachen, später mehr}

		\item Subclassing
		\note[item]<3> {Programmiersprache muss neues Konzept subclassing beinhalten}
		\note[item]<3> {Der Versuch, beide Konzepte unter einen Hut zu bringen}
		\note[item]<3> {Auch hier später mehr}

		\item Übersetzung in eine andere Sprache
		\note[item]<4> {Nicht nur Theorie sondern auch Praxis}
		\note[item]<4> {Übersetzung in andere höhere Sprache}

		\item Matching
		\note[item]<5> {Ursprünglich auch vorgesehen: matching}

	\end{bigitemize}

	\note{Heine}
\end{frame}

\subsection{Benennung}
\begin{frame}[t]{Benennung}
	\begin{bigitemize}[<+->]{3mm}

		\item Ursprünglicher Projektname: OOPLSS
		\note[item]<1>{Ursprünglicher Name}
		\note[item]<1>{Object-oriented language with subtyping and sublassing}

		\item Name der Sprache: LISA 
		\note[item]<2>{Name der Sprache}
		\note[item]<2>{Rekursives Akronym für Lisa includes Subclassing}

	\end{bigitemize}


	\note{Heine}
\end{frame}

\subsection{Ziele}
\begin{frame}[t]{Ziele}

	\begin{bigitemize}[<+->]{3mm}
		\item Neue Lerninhalte
		\note[item]<1> {Andere Richtung einschlagen als bisher an der Schule}
	\end{bigitemize}

	\note{Heine}
\end{frame}
