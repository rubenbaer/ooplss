\section{Einleitung}
\subsection{Aufgabenstellung}
\begin{frame}[t]{Aufgabe und Ziel}
	\begin{itemize}[<+->]
		\item Aufgabenstellung
		\begin{itemize}
			\item Programmiersprache
			\note[item]<2> {Eine objektorientierte Programmiersprache entwickeln}

			\item Subtyping
			\note[item]<3> {Programmiersprache muss Konzept subtyping beinhalten}
			\note[item]<3> {Bekannt bei aktuellen Sprachen, später mehr}

			\item Subclassing
			\note[item]<4> {Programmiersprache muss neues Konzept subclassing beinhalten}
			\note[item]<4> {Der Versuch, beide Konzepte unter einen Hut zu bringen}
			\note[item]<4> {Auch hier später mehr}

			\item Übersetzung in eine andere Sprache
			\note[item]<5> {Nicht nur Theorie sondern auch Praxis}
			\note[item]<5> {Übersetzung in andere höhere Sprache}

			\item Matching
			\note[item]<6> {Ursprünglich auch vorgesehen: matching}
		\end{itemize}

		\item Ziele
		\begin{itemize}
			\item Sprache nur als Prototyp
			\note[item]<8> {Sprache ist nur als Prototyp gedacht}
			\note[item]<8> {Als eine Art Spielsprache}
			\item Einsicht in die Welt des Compilerbau
			\note[item]<9> {Neue Lerninhalte}
		\end{itemize}

	\end{itemize}

	\note{Heine}
\end{frame}

\subsection{Benennung}
\begin{frame}[t]{Benennung}
	\begin{bigitemize}[<+->]{3mm}

		\item Ursprünglicher Projektname: OOPLSS
		\note[item]<1>{Ursprünglicher Name}
		\note[item]<1>{Anfang der Thesis}
		\item Object oriented programming language with subtyping and subclassing
		\note[item]<1>{Object-oriented language with subtyping and subclassing}

		\item Name der Sprache: LISA 
		\note[item]<3>{Name der Sprache}
		\note[item]<3>{Ooplss etwas mühsam zum ausprechen}
		\item \textbf{L}isa \textbf{i}ncludes \textbf{S}ubcl\textbf{a}ssing
		\note[item]<4>{Rekursives Akronym für Lisa includes Subclassing}

	\end{bigitemize}


	\note{Heine}
\end{frame}

%\subsection{Ziele}
%\begin{frame}[t]{Ziele}
%
%	\begin{bigitemize}[<+->]{3mm}
%		\item Neue Lerninhalte
%		\note[item]<1> {Andere Richtung einschlagen als bisher an der Schule}
%	\end{bigitemize}
%
%	\note{Heine}
%\end{frame}
