\section{Einleitung}

\begin{frame}[t]{Aufgabenstellung}
	\begin{itemize}[<+->]
	\item Programmiersprache
	\note[item]<2> {Eine objektorientierte Programmiersprache entwickeln}

	\item Subtyping
	\note[item]<3> {Programmiersprache muss Konzept subtyping beinhalten}
	\note[item]<3> {Bekannt bei aktuellen Sprachen, später mehr}

	\item Subclassing
	\note[item]<4> {Programmiersprache muss neues Konzept subclassing beinhalten}
	\note[item]<4> {Der Versuch, beide Konzepte unter einen Hut zu bringen}
	\note[item]<4> {Auch hier später mehr}

	\item Übersetzung in eine andere Sprache
	\note[item]<5> {Nicht nur Theorie sondern auch Praxis}
	\note[item]<5> {Übersetzung in andere höhere Sprache}

	\item Matching
	\note[item]<6> {Ursprünglich auch vorgesehen: matching}

	\end{itemize}

	\note{Heine}
\end{frame}

\begin{frame}[t]{Benennung}
	\begin{itemize}[<+->]
		\item Ursprünglicher Projektname: OOPLSS
		\note[item]<1>{Ursprünglicher Name}
		\note[item]<1>{Object-oriented language with subtyping and sublassing}

		\item Name der Sprache: LISA 
		\note[item]<2>{Name der Sprache}
		\note[item]<2>{Rekursives Akronym für Lisa includes Subclassing}

	\end{itemize}


	\note{Heine}
\end{frame}

\begin{frame}[t]{Ziele}

	\begin{itemize}[<+->]
		\item Neue Lerninhalte
	\end{itemize}

	\note{Heine}
\end{frame}
